\section{Notations  and preliminaries}\label{defi:sec}
 %\todo{BK: Maybe we can shorten this a bit. What do you think?
 %\\L: Yes, why do we need two definitions of a path? I commented out the one as a sequence of %vertices.}
 Given a digraph $G=(V,A)$, a simple directed $s$-$t$-path $P$ in $G$ is specified as a sequence of
arcs $P=(a_1,a_2,\ldots,a_p)$ such that  $a_1$ starts at $s$, $a_p$
ends at $t$, nonconsecutive arcs do not share a vertex and  the end vertex of $a_{i}$ coincides with  the start vertex of $a_{i+1}$ for any $i\in \{1,\ldots,p-1\}$. The number
$p$ of arcs in $P$ is called the length of the path. We sometimes use the same notation for a path $P$ and the  set of its arcs. 
%Alternatively, a simple directed $s$-$t$-path $P$ of length $p$ in $G$  is specified as a sequence of pairwise disjoint vertices $(u_0,u_1,\dots,u_p)$, 
%where $x_0=s$, $u_p=t$ and $(u_i,u_{i+1})\in A$ for all $i\in\{0,\dots,p\}$. Thus  $a_i=(u_{i-1},u_i)$, for $i\in \{1,\dots,p\}$. 
We  consider a single arc $(x, y)$ as an $x$-$y$-path of length $1$  and a single vertex $x$ as a  trivial $x$-$x$-path of length $0$. Given an  $x$-$y$-path $P_1$ and a  $y$-$z$-path $P_2$, we denote the \emph{concatenation} of $P_1$ and $P_2$ by $P_1 \cdot P_2$. We also consider concatenations of paths and arcs, that is, terms of the form $P \cdot a$ for some $x$-$y$-path $P$ and some arc $a = (y, z)$.
 
In the linearization problem, we are concerned with digraphs $G =(V, A)$ with a source vertex $s$ and a sink vertex $t$. 
We denote by  $\Pst$ the set of all simple directed $s$-$t$-paths.
%\todo{BK Reformulated. Before it said all paths from s to t. Check!}
%all paths from $s$ to $t$. 
%\todo{L: Added explanation, why we can assume $\Pst$-covered w.l.o.g.\\ SL: I think we %should be careful with saying it like that. 
%Checking if a graph is $\Pst$-covered is NP-hard, right?.
%,right=
%\\R:  Not in an acyclic graph.
%In an acyclic graph the existence of a simple directed s-t-path using an arc (x,y) 
%is equivalent to the existence of a simple directed  s-x path and a simple directed %y-t-path. If the two  paths exist, they will be vertex-disjoint due to acyclicity.
%BK: Shortened R's todo answer.}

We often assume that $G$ is \emph{$\Pst$-covered}, that is, every arc in $G$ is traversed by at least one path in $\Pst$. 
Note that we can make this assumption without loss of generality for acyclic graphs: If some arc is not traversed by at least one $s$-$t$-path (which can be decided in polynomial time for acyclic graphs), then it has no effect on the linearizability of the instance, and so we can delete that arc. 

Let $d \geq 2$ be a natural number. The \emph{Order-$d$ interaction costs} are given by a mapping $q_d\colon \set{B \subseteq A : |B| \leq d}\to \rz$, assigning a (potentially negative) interaction cost to every subset of at most $d$ arcs.
The cost $\text{SPP}_d(P,q_d)$ of some path $P$ under interaction costs $q_d$ is defined as in equation (\ref{dSPP:obj}).
If $d$ is unambiguously clear from the context, we use the  more compact notation $f_q(P): = \text{SPP}_d(P,q_d)$.
 In this chapter we explicitly allow the case $q(\emptyset) \neq 0$, because this simplifies the calculations.  
The \emph{linearization problem for the Order-$d$ Shortest Path Problem} (\textsc{Lin}SPP$_d$)
is formally defined as follows.
\begin{center}
%%%%%%%%%%%%%%%%%
\boxxx{\textbf{Problem:} The  linearization problem for the SPP$_d$ (\textsc{Lin}SPP$_d$)
\\[1.0ex]
\textbf{Instance:} A $\Pst$-covered directed graph $G=(V,A)$ with $s,t\in V$, $s\neq t$; an integer $d \geq 2$;
an order-$d$ arc interaction cost function $q_d: \set{B \subseteq A \colon |B| \leq d} \to \R$.
\\[1.0ex]
\textbf{Question:} Find a \emph{linearizing cost function}  $c\colon A\to\R$  such that $\text{SPP}_d(P,q_d) = \text{SPP}(P,c)$ for all $P \in \Pst$ or decide that such a linearizing  cost function does not exist.}
%%%%%%%%%%%%%%%%%
\end{center}
%%%%%%%%%%%%
In the special case $d = 2$, we obtain the    linearization problem for the QSPP (\textsc{Lin}QSPP).
\smallskip

Finally, let us consider the \emph{Generic Shortest Path Problem} (GSPP) which takes as input  a 
%\todo{R: I am adding ``acyclic'' here, to avoid  problems with the w.l.o.g. assumption about $G$ being ${\cal P}_{st}$-covered  (see the previous todo-remark)
 digraph $G=(V,A)$ with a source vertex $s$, a sink vertex $t$, $s\neq t$, and a generic cost function\footnote{We assume that $f$ is specified by an oracle.} $f\colon \Pst \to \rz$  assigning  a cost $f(P)$ to every path $P\in \Pst$. 
%We assume w.l.o.g.\ that $G$ is $\Pst$-covered. 
The goal is to find an $s$-$t$-path  which minimizes the objective function $f(P)$ over $\Pst$. 
A linearizable instance of  the GSPP and the linearization problem for the GSPP (\textsc{Lin}GSPP) are  defined analogously as in the respective definitions for SPP$_d$.


%\todo{L: Added 
%this 
%lemma. 
%(we had an email exchange about it). 
%Do you find it relevant enough 
%to include
%? 
%SL: There is no lemma here anymore. Is this comment still relevant?
We note that in our \cref{def:linearizable} we allow the linearizing function $c : A \to \R$ to  take negative values. For graphs with negative cycles, this could create an issue. Cycles are excluded however in acyclic graphs to which the results in this chapter apply to.
%The absence of  negative cycles is guaranteed by the assumption that the considered graph >$G$ is acyclic. 
%In fact, the following lemma shows that for acyclic digraphs it is a more difficult problem to decide if there is a linearizing function $c : A \to \R$ than the problem to decide if there is a linearizing function $c': A \to \R_+$. Hence, we consider only the first problem for the rest of the chapter.
Note that for acyclic graphs
the decicion problem  about the existence of a nonnegative linearizing function can be reduced to the decision problem about the existence of a real-valued  linearizing function, as shown by the following lemma.  
\begin{lemma}
If $(G,s,t,f)$ is an instance of the GSPP such that $G  = (V,A)$ is an acyclic digraph, then there is a nonnegative linearizing function $c' : A \to \R_+$ if and only if $f(P) \geq 0$ for all $s$-$t$-paths $P$ and there is a linearizing function $c : A \to \R$.
\end{lemma}
\begin{proof}
If there is no linearizing function $c : A \to \R$, then there is also no linearizing function $c' : A \to \R_+$.

Assume now that 
there exists  a linearizing function $c : A \to \R$. Clearly, if  there is an $s$-$t$-path $P$ with $f(P) < 0$, then no nonnegative linearizing function exists. 
 If  $f(P) \geq 0$ for all $s$-$t$-paths,  a nonnegative linearizing function  $c'$ can be constructed by using the reduced costs $c_\pi(u,v) := \pi(u) - \pi(c) + c(u,v)$, for $(u,v) \in A$, where $\pi(v)$ denotes  the cost of a shortest $s$-$v$-path with respect to $c$, for $v\in V$.  We set  $c'(a) := c_\pi (a) + \pi(t)$ for all arcs incident to the source and $c'(a) := c_\pi(a)$ otherwise.  Since $\pi(t) \geq 0$ (due to the assumption that $f(P)\ge 0$ for all $s$-$t$-paths) and $c_\pi(u,v)\ge 0$, for all $(u,v) \in A$ (due to the acyclicity of $G$), $c'$  takes only nonnegative values.    Moreoever, for all $s$-$t$-paths $P$ the equalities  $c_\pi(P) = c(P) + \pi(s) - \pi(t) = c(P) - \pi(t)$ and $c'(P)=c_\pi(P)+\pi(t)$ hold, implying   $c'(P)=c(P)=f(P)$.
\end{proof}
