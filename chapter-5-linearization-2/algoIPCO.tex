\section{A linear time algorithm for the \textsc{Lin}SPP$_d$}\label{algo:sec}
%%%%%%%%%%%%%%%%%
In this section, we describe an algorithm which solves the linearization problem for SPP$_d$ (\textsc{Lin}SPP$_d$) in  $\bigO(m^d)$ time, i.e., in linear time with respect ot the input size.
The algorithm uses the relationship between the \textsc{Lin}SPP$_d$ and the  \emph{All-Paths-Equal-Cost Problem (APECP)} which we introduce in  Section~\ref{APEC:ssec}.  
In \cref{subsection:alg} we describe the SPP$_d$ algorithm and discuss
its running time.

 \subsection{The All Paths Equal Cost Problem of Order-$d$ (APECP$_d$)}
\label{APEC:ssec}

The All Paths  Equal Cost Problem of Order-$d$ (APECP$_d$) is defined as follows.
\begin{center}
%%%%%%%%%%%%%%%%%
\boxxx{\textbf{Problem:}  ALL PATHS EQUAL COST of Order-$d$ (APECP$_d$)
\\[1.0ex]
\textbf{Instance:} An acyclic $\Pst$-covered directed graph $G=(V,A)$ with a source vertex $s$ and a sink vertex $t$, an  integer $d \geq 1$;
an order-$d$   cost function $q_d\colon  \set{B \subseteq A : |B| \leq d} \to \R$.
\\[1.0ex]
\textbf{Question:} Do all $s$-$t$-paths  have the same cost, i.e.\ is there some $\beta \in \R$ such that $\text{SPP}_d(P,q_d) = \beta$ for every path $P$ in $\Pst$?}
%%%%%%%%%%%%%%%%%
\end{center}

In the following we establish a connection between the \textsc{Lin}SPP$_d$ and the APECP$_{d-1}$ for $d\ge 2$. More precisely, we show in Lemma~\ref{lemma:corresponding-instance}  that    an   instance $(G,s,t,q_d)$ of the  \textsc{Lin}SPP$_d$ with an acyclic  $\Pst$-covered digraph $G=(V,A)$ can  be reduced  to $\bigO(m)$ instances of APECP$_{d-1}$, each of them corresponding to exactly one strongly basic arc with respect to some fixed system of nonbasic arcs (see Definitions~\ref{nonbasic:def} and \ref{stronglybasic:def}).  The  APECP$_{d-1}$ instance corresponding to a strongly basic arc  $a = (u, v)$ is defined as follows.
\begin{definition}
\label{def:corresponding-instance}
Let $d \geq 2$ and let  $(G, s,t, q_d)$ be an instance of the \textsc{Lin}SPP$_d$ with an acyclic  $\Pst$-covered digraph $G=(V,A)$ and a fixed system of nonbasic arcs  $N$.   
Let $a=(u,v)$ be a  strongly basic arc. Denote by   $V_u\subseteq V$  the set of vertices  lying on at least one $s$-$u$-path and  by  $A_u\subseteq A$  the set of arcs lying on at least one $s$-$u$-path. Set  $G^{(a)}: = (V_u, E_u)$. 
    The   instance $I^{(a)}$ of the APECP$_{d-1}$ corresponding to the strongly basic arc $a$ is specified 
    by the digraph $G^{(a)}$, the source vertex $s(a):=s$, the sink vertex $t(a):= u$ and the  order-$(d-1)$  cost function   $q_{d-1}^{(a)}\colon \{ B \subseteq A_u\colon  |B| \leq d-1\} \to \rz$  defined by means of the function $q_d$ as follows:  \begin{comment}
  The instance $I^{(a)}$ of the APECP$_{d-1}$ corresponding to the strongly basic arc $a=(u,v)$ takes as input   the digraph $G^{(a)} = (V_u, E_u)$ with  source vertex $s' = s$,  sink vertex $t' = u$, where 
 $V_u$ is the set of vertices in $V$ lying on at least one $s$-$u$-path and $A_u$  is the set of arcs in $A$ lying on at least one $s$-$u$-path.
 
  The  order-$(d-1)$  cost function   $q_{d-1}^{(a)}\colon \{ B \subseteq A_u\colon  |B| \leq d-1\} \to \rz$ is given by
  \end{comment}
    \begin{equation} \label{eq:definition_q_e}
        q_{d-1}^{(a)}(B) := \left(\sum_{\substack{C \subseteq N_u\\ |C| \leq d - |B|}}q_d(B \cup C) \right) - \left( \sum_{\substack{C \subseteq a \cdot N_v\\ |C| \leq d - |B|}}q_d(B \cup C) \right). 
    \end{equation}
\end{definition}

\begin{lemma}
\label{lemma:corresponding-instance}
Let $d \geq 2$ and let  $(G, s,t, q_d)$ be an instance of the \textsc{Lin}SPP$_d$ with an acyclic  $\Pst$-covered digraph $G=(V,A)$ and a fixed system of nonbasic arcs  $N$. 
The APECP$_{d-1}$ instance $I^{(a)}$ corresponding to some strongly basic arc $a$ is a \textsc{YES}-instance iff the arc $a$ has the property $(\pi)$ with respect to $f\colon \Pst\to\rz$ given by $f(P)=SPP_d(P,q_d)$ for $P\in \Pst$. In this case, $\val(a) = \beta_a$, where $\beta_a$ is the common cost of all source-sink-paths in the APECP$_{d-1}$ instance $I^{(a)}$.
\end{lemma}
\begin{proof}
Let $a = (u,v) \in A$ be a strongly basic arc and let $P$ be some $s$-$u$-path in $G$. Then $P$ is  contained in the graph $G^{(a)} = (V_u, A_u)$. 
Denote $f^{(a)}(P):=SPP_{d-1}(P,q^{(a)}_{d-1})$ for any $s$-$u$-path $P$ in $G$. We have that
\begin{align*}
    &\quad \val(a, P) = f_q(P \cdot N_u) - f_q(P \cdot a \cdot N_v) \\
    &= \sum_{\substack{F \subseteq P \cdot N_u\\ |F| \leq d}}q_d(F) - \sum_{\substack{F \subseteq P \cdot a \cdot N_v\\ |F| \leq d}}q_d(F)\\
    &= \sum_{k=0}^d \sum_{\substack{B \subseteq P\\ |B| = k}}\sum_{\substack{C \subseteq N_u\\ |C| \leq d - k}}q_d(B \cup C) - \sum_{k=0}^d \sum_{\substack{B \subseteq P\\ |B| = k}}\sum_{\substack{C \subseteq a \cdot N_v\\ |C| \leq d - k}}q_d(B \cup C)\\
    &= \sum_{k=0}^{d-1} \sum_{\substack{B \subseteq P\\ |B| = k}}\sum_{\substack{C \subseteq N_u\\ |C| \leq d - k}}q_d(B \cup C) - \sum_{k=0}^{d-1} \sum_{\substack{B \subseteq P\\ |B| = k}}\sum_{\substack{C \subseteq a \cdot N_v\\ |C| \leq d - k}}q_d(B \cup C) \quad +  (1 - 1)\sum_{\substack{B \subseteq P\\ |B| = d}}q_d(B)\\
     &= \sum_{k=0}^{d-1} \sum_{\substack{B \subseteq P\\ |B| = k}}\left(\sum_{\substack{C \subseteq N_u\\ |C| \leq d - k}}q_d(B \cup C) - \sum_{\substack{C \subseteq a \cdot N_v\\ |C| \leq d - k}}q_d(B \cup C)
     \right)\\
     &= \sum_{\substack{B \subseteq P\\ |B| \leq d-1}}q^{(a)}_{d-1}(B) = f^{(a)}(P).
\end{align*}
We conclude that the value $\val(a, P)$ is independent of $P$, if and only if for every path the quantity $f^{(a)}(P)$ does not depend on $P$. The latter condition is equivalent to  $I^{(a)}$  being a \textsc{YES}-instance of the APECP$_{d-1}$. Furthermore, if this is the case, then $\val(a) =\beta_a= f^{(a)}(P)$ for any $s$-$u$-path $P$.
\end{proof}
\cref{lemma:property-pi,lemma:corresponding-instance} imply that an instance $(G,s,t,q_d)$ of the SPP$_d$ with an acyclic digraph $G$ is linearizable  iff each  instance $I^{(a)}$ of the APECP$_{d-1}$ corresponding to some strongly basic arc $a$   (with respect to some fixed system of nonbasic arcs)  is a YES-instance. Furthermore, the same lemmas imply that in this case a linearizing cost function in reduced form is obtained by letting $c(a) = \beta_a$ for all strongly basic arcs, $c(a) = 0$ for all nonbasic arcs, and $c(a) = \text{SPP}_d(a \cdot N_v)$ for all arcs $a = (s,v)$ incident to the source.


Thus, we have shown that an instance of the  \textsc{Lin}SPP$_d$ can be reduced to $\bigO(m)$ instances  of the APECP$_{d-1}$.  
Next, in Lemma~\ref{lemma:reduction-APEC} we show that each instance of the APECP$_{d-1}$  can  be reduced to an instance of  the \textsc{Lin}SPP$_{d-1}$.
%
First, we define a specific cost function as follows. 
Let $G = (V, A)$ be a $\Pst$-covered acyclic digraph and $\beta \in \R$. The function $\const_\beta : A \rightarrow \R$  assigns cost $\beta$ to every arc incident to the source $s$, and $0$ to all other arcs. 

\begin{lemma}
\label{lemma:reduction-APEC}
    Let $G = (V, A)$ be a $\Pst$-covered acyclic digraph with source vertex $s$ and sink vertex $t$ and let  $N \subseteq A$ a fixed system of nonbasic arcs. Let $q_d$ be an order-$d$  cost function. The instance $(G,s,t,q_d)$ of the APECP$_d$ problem is a YES-instance iff the instance $(G,s,t, q_d)$ of SPP$_d$ is linearizable and $\const_\beta$ is its unique linearizing function in reduced form (with respect to $N$).
\end{lemma}
\begin{proof}
    Clearly,  $\const_\beta$ is a linearizing function iff   all paths have the same cost $\beta$. % Further, if all paths have the same cost $\beta$, then  $\const_\beta$ is a linearizing function. 
    Furthermore, observe that all  arcs incident to the source do not belong to $N$. Therefore $\const_\beta$ is in reduced form with respect to $N$. In fact, by \cref{lemma:nonbasic-arcs}  $\const_\beta$ is the unique linearizing  functions in reduced form, and $\red(c') = \const_\beta$ for all other linearizing functions $c'$.
\end{proof}


Finally, we show in %\cref{lemma:reduction-APEC} 
\cref{lemma:apec-1} 
 how to solve the APECP$_1$ and how to  compute the common cost of all source-sink-paths in the case of a  YES-instance.
\begin{lemma}
\label{lemma:apec-1}
    The APECP$_1$ can be solved in linear time $\bigO(m)$.
\end{lemma}
\begin{proof}
    This is an easy exercise in dynamic programming. The algorithm uses the fact that in a $\Pst$-covered acyclic digraph all $s$-$t$-paths have the same cost if and only for every vertex $w$ all $s$-$w$-paths have the same cost. Hence we can introduce a variable $y_w \in \R$ for every vertex $w$. We let $y_s = 0$ and then check in topological vertex order for every vertex $w$, whether the value $y_u + q_1(\set{(u,w)})$ is the same for every incoming edge $(u, w)$. Finally, if this is the case, the  common cost of all source-sink paths is given by $y_t + q_1(\emptyset)$.
\end{proof}
 We remark that attention to the case $q_1(\emptyset) \neq 0$ is crucial. This is because by definition, the term $q_1(\emptyset)$ is always included in SPP$_1(P, q_1)$ for every possible path $P$. Indeed, \cref{eq:definition_q_e} may produce instances with the property $q_1(\emptyset) \neq 0$.

%%%%%%%%%%%%%%%%%%%%%%%%%%%%%%%%%%%%%%%%%%%%%%%%%%%%%%%%%%%%%%%%%%%%%%%%%%%

\subsection{The linear time \textsc{Lin}SPP$_d$ algorithm}
\label{subsection:alg}
%\label{subsection:alg-description}
%%%
%An important part of the algorithm is solving the \emph{All-Paths-Equal-Cost Problem (APEC)}.
% \begin{corollary}
% An acyclic digraph with given arc interaction costs is linearizable if and only if (with respect to some fixed system of nonbasic arcs) for each strongly basic arc its corresponding APEC instance is a YES-instance.
% \end{corollary}

% \begin{proof}
%     This follows immediately from \cref{lemma:property-pi,lemma:corresponding-instance}.
% \end{proof}
Our \textsc{Lin}SPP$_d$ algorithm ${\cal A}$ works as follows.
Consider an instance $(G,s,t,q_d)$ of the \textsc{Lin}SPP$_d$ with an acyclic $\Pst$-covered digraph $G$, with source vertex $s$, sink vertex $t$ and order-$d$ cost function $q_d$. We first fix some system of nonbasic arcs $N$ and construct the instance $I^{(a)}$ of the  APECP$_{d-1}$ problem given in \cref{def:corresponding-instance} for each strongly basic arc $a$. 
Then,  we  check each instance  $I^{(a)}$ for being a  \textsc{YES}-instance and
do this by reducing  $I^{(a)}$  to  an instance of \textsc{Lin}SPP$_{d-1}$ according to \cref{lemma:reduction-APEC}.
By iterating this process we eventually end up with instances of APECP$_1$  which  can be easily solved by  dynamic programming, as demonstrated in \cref{lemma:apec-1}. A summary in pseudocode is provided in Algorithm~\ref{alg}. The correctness of the algorithm follows from \cref{lemma:property-pi,lemma:corresponding-instance,lemma:reduction-APEC,lemma:apec-1}.


\begin{algorithm}[htpb]
    \SetKwFunction{FLinearizable}{Linearizable}
    \SetKwFunction{FAPEC}{APECP}
    \SetKwProg{Fn}{Function}{:}{}
 \Fn{\FLinearizable{$G$,$s$,$t$,$q$,$d$}}{
 \KwIn{acylic $\Pst$-covered $G$; source $s$; sink $t$; integer $d \geq 2$; order-$d$ interaction costs $q_d$.}
 \KwOut{$\False$ if %$(G, q_d)$ 
 is not linearizable, otherwise a tuple $(\True, c)$ such that $c$ is a linearizing cost function in reduced form.}
 $N \gets$ some system of nonbasic arcs\;
 Calculate $\set{(G^{(a)},q^{(a)}_{d-1}) : a \text{ strongly basic}}$ \tcp*{using \cref{lemma:faster-reduction}.} \label{alg:linenumber:3}
 \eIf{\FAPEC{$G^{(a)}$,$s(a)$,$t(a)$,$q^{(a)}_{d-1}$,$d-1$} = {\False} for some strongly basic $a$}{
    \KwRet \False\;
 }{
    \FAPEC{$G^{(a)}$,$s(a)$,$t(a)$,$q^{(a)}_{d-1}$, $d-1$}=$(\True, \beta_a)$  for all strongly basic $a$\\
    $c(a) \gets
    \begin{cases}
        \beta_a; & \text{$a$ strongly basic. }\\
        f_q(a \cdot N_v) & \text{$a = (u,v)$ incident to source} \label{alg:linenumber:f_q} \tcp*{using \cref{lemma:faster-reduction}.}\\
        0; &  \text{$a$ nonbasic.}
    \end{cases}$\;
    \KwRet $(\True, c)$\;
 }
 }
 
 \Fn{\FAPEC{$G$,$s$,$t$,$q$,$d$}}{
  \KwIn{acylic $\Pst$-covered $G$; source $s$; sink $t$; integer $d \geq 1$; order-$d$ interaction costs $q_d$.}
 \KwOut{$\False$ if not all $s$-$t$-paths in $(G, q_d)$ have the same cost, otherwise a tuple $(\True, \beta)$ such that all paths have cost $\beta$.}
 \uIf{$d=1$}{
    solve by applying \cref{lemma:apec-1}
 }
 \uElseIf{\FLinearizable{$G$,$q_d$,$d$} = \False}{
    \KwRet \False\;
 }
 \uElse{
    \FLinearizable{$G$,$q_d$,$d$} = $(\True, c)$ for some $c$\;
    \eIf{$c = \const_\beta$ for some $\beta$}{
        \KwRet $(\True, \beta)$\;
    }{
        \KwRet \False\;
    }
 }
 }
 \caption{An algorithm to solve the \textsc{Lin}SPP$_d$.}
 \label{alg}
\end{algorithm}


\label{subsection:alg-runtime}
It is not hard to implement the algorithm described above
in $\bigO(d^2m^{d+1})$ time. %However, we can do better. 
With a careful implementation it is possible to achieve a better result as stated in \cref{thm:alg-linear-time}.

\begin{theorem}
\label{thm:alg-linear-time}
The \textsc{Lin}SPP$_d$ on acyclic digraphs can be solved in $\bigO(d^2m^d)$ time.
\end{theorem}

Note that  the input size required to encode the cost function $q_d$ equals $\sum_{k=0}^d \binom{m}{k} \geq \frac{m^d}{d!}$. Thus, $\bigO(d^2m^d)$ is linear in the input size and hence  optimal if  $d$ is considered a constant, like for example in the QSPP.
The remainder of the section is devoted to the proof of \cref{thm:alg-linear-time}. The main bottleneck we have to get rid of is the computation of the instances $I^{(a)}$ corresponding to the strongly basic arcs $a$ in line~\ref{alg:linenumber:3} of Algorithm~\ref{alg}. If one simply uses the definition of $q^{(a)}_{d-1}$ from \cref{eq:definition_q_e}, then one can see that for each $a$ one can compute $q^{(a)}_{d-1}$ in $\bigO(dm^d)$ time. As this needs to be repeated for each strongly basic arc $a$, in total this would take $\bigO(dm^{d+1})$ time. A similar bottleneck arises in line~\ref{alg:linenumber:f_q} when computing $f_q(a \cdot N_v)$. We get rid of these two bottlenecks by calculating some %\emph{helper values}
\emph{auxiliary values}. 

For all sets $B \subseteq A$ of arcs with $|B| \leq d-1$ and all vertices $x \in V \setminus \set{s}$, we define the  value
\begin{equation}
    \gamma(B, x) := \sum_{\substack{C \subseteq N_x\\ |C| \leq d - |B|}}q_d(B \cup C). \label{eq:helper-values}
\end{equation}
 
\begin{lemma}
\label{lemma:faster-reduction}
The line~\ref{alg:linenumber:3} and the line~\ref{alg:linenumber:f_q} of Algorithm~\ref{alg} can be implemented such that their execution takes at most $c d m^d$ steps for some constant $c \geq 0$, independent of both $n$ and $d$.
\end{lemma}
\begin{proof}
    Consider the auxiliary  values $\gamma(B,x)$ from \cref{eq:helper-values}.
    We show how to compute all the values of $\gamma$ for $B \subseteq A, |B| \leq d-1, x \in V \setminus \set{s}$ in $\bigO(dm^d)$ time. We begin by showing how to compute the value $\gamma(B, x)$ for all vertices $x \in V\setminus\set{s}$ and  for some fixed set $B \subseteq A$ with $|B| \leq d-1$. Let $k = |B|$ be the size of $B$,  $k \in \fromto{0}{d-1}$. Assume that $\gamma(B, v)$ has  already been computed for some nonbasic arc $a = (u, v) \in N$. Then,  $\gamma(B,u)$ can be computed as follows.  Since  $a \in N$, we have $N_u = a \cdot N_v$. Moreover, the following equalities hold 
    \begin{align}
        \gamma(B, u) &= \sum_{\substack{C \subseteq a \cdot N_v\\ |C| \leq d - |B|}}q_d(B \cup C) \nonumber\\
        &=  \sum_{\substack{C \subseteq N_v\\ |C| \leq d - |B| - 1}}q_d(B \cup \set{a} \cup C) + \sum_{\substack{C \subseteq N_v\\ |C| \leq d - |B|}}q_d(B \cup C)\nonumber\\
        &=  \gamma(B, v) + \sum_{\substack{C \subseteq N_v\\ |C| \leq d - |B| - 1}}q_d(B \cup \set{a} \cup C). \label{eq:recursive-gammma}
    \end{align}
    This formula can be evaluated in $\binom{m}{{d - k - 1}}$ time. We can now traverse the tree of nonbasic arcs, starting at the root $t$, where $\gamma(B, t) = q(B)$, and iteratively apply the formula (\ref{eq:recursive-gammma}) to compute  all  values $\gamma(B, x)$ for the vertices $x \in V \setminus \set{s}$.
    Thus,  it takes $n\binom{m}{{d - k - 1}}$ time to compute all values $\gamma(B, x)$ for a fixed set $B$ of size $k$. For each $k = 0,\dots,d-1$, there are $\binom{m}{k}$ subsets of size $k$. Therefore, the total time to compute all values of $\gamma$ is
    \begin{align*}
        \sum_{k=0}^{d-1} n\binom{m}{d - k - 1}\binom{m}{k} \leq \sum_{k=0}^d n\frac{m^{d-k-1}}{(d-k-1)!} \frac{m^k}{k!} \leq (d+1) nm^{d-1}.
    \end{align*}
    (Note that here we used the very rough estimation $t! \geq 1$.) As the digraph is $\Pst$-covered, we can assume w.l.o.g.\ that it is connected, implying  $n\le 2m$ and  consequently,    ($d+1) nm^{d-1}\le 4dm^d$. Thus,  there exists a  constant $c_1 \geq 0$, independent on $n$ and $d$,  such that all needed values of $\gamma$ can be computed in $c_1dm^d$ time. 

    
    Now assume that all the values $\gamma(B, x)$ have been computed. For every strongly basic arc $a = (u, v)$, and each set $B \subseteq A_u$ with $|B| \leq d-1$, the order-$d-1$ cost function in the corresponding APECP$_{d-1}$ instance fulfills 
    \begin{align*}
        q_{d-1}^{(a)}(B) = \gamma(B,u) - \gamma(B, v) - \sum_{\substack{C \subseteq N_v\\ |C| \leq d - |B| - 1}}q_d(B \cup \set{a} \cup C).
    \end{align*}
    This can be seen by plugging the definition of $\gamma$ into \cref{eq:definition_q_e}. This equation can be evaluated in $\binom{m}{d - k -1}$ time if $k=0,\dots,d-1$ is the size of $B$. In order to obtain all the desired values for $q^{(a)}_{d-1}(B)$ for all the APECP$_{d-1}$ instances, we need to consider every choice of the set $B$ (again there are at most $\binom{m}{k}$ sets of size $k$) and every of the at most $\bigO(m)$ choices of the arc $a$. 
    The same analysis as before shows that given the computed values of  $\gamma$, all required values of $q^{(a)}_{d-1}(B)$ can be computed in at most $c_3dm^d$ steps for some constant $c_3 \geq 0$. Finally, computing the vertex set $V_u$ and arc set $A_u$ of the APECP$_{d-1}$ instance clearly can be done in linear time $\bigO(m)$ for each strongly basic arc $a = (u,v)$. We conclude that computing the set $\set{(G^{(a)}, q^{(a)}_{d-1}) : a \text{ strongly basic })}$ of digraphs for  all APECP$_{d-1}$ instances can be done in at most $c_4dm^d$ steps for some constant $c_4 \geq 0$.

    Finally consider line~\ref{alg:linenumber:f_q} of Algorithm~\ref{alg} and assume that all auxiliary  values $\gamma(B,x)$ have been computed. Let $a = (s,v)$ be an arc incident to the source. We have that
    \begin{align*}
        f_q(a \cdot N_v) = \gamma(\emptyset, v) + \sum_{\substack{C \subseteq N_v\\ |C| \leq d - 1}}q(\set{a} \cup C).
    \end{align*}
    This formula can be evaluated in $\bigO(m^{d-1})$ time for each arc $a$. We conclude that the set of all values $f_q(a \cdot N_v)$ in line~\ref{alg:linenumber:f_q} of Algorithm~\ref{alg} can be computed in at most $c_5m^d$ steps for some constant $c_5 \geq 0$, thus completing the proof of the lemma.
\end{proof}

\begin{proof}[Proof of \cref{thm:alg-linear-time}]
It follows from \cref{lemma:property-pi,lemma:corresponding-instance,lemma:reduction-APEC,lemma:apec-1} that Algorithm~\ref{alg} correctly solves the \textsc{Lin}SPP$_d$. It follows from \cref{lemma:faster-reduction} that lines \ref{alg:linenumber:3} and \ref{alg:linenumber:f_q} can be implemented to take at most $c_0dm^d$ steps each for some constant $c_0 \geq 0$. The running time of all the remaining lines is asymptotically dominated by $m^d$. Now let $f(m, d)$ denote the worst-case running time of $\FLinearizable{}$ on an instance with  $m$ arcs and an order-$d$  cost function.  Let $g(m, d)$ be the corresponding function for $\FAPEC{}$. By the preceding arguments, there exists a constant $c \geq 0$ such that the following inequalities hold:
\begin{align*}
    f(m ,d) &\leq cdm^d + mg(m, d-1); & d \geq 2\\
    g(m, d) &\leq cm + f(m, d); & d \geq 2\\
    g(m, 1) &\leq cm.
\end{align*}
By induction over $d$ it is easy ti see that $f(m, d) \leq cd(2d - 2)m^d$ and $g(m, d) \leq cd(2d - 1)m^d$. Indeed, in the base case $d = 1$ we have $g(m, 1) \leq cm$. For $d \geq 2$ we have 
\begin{align*}
    f(m, d) &\leq cdm^d + mc(d-1)(2d - 3)m^{d-1} \leq cd(2d - 2)m^d \\
    g(m, d) &\leq cm + cd(2d - 2)m^d \leq cd(2d -1)m^d. 
\end{align*}
This implies that $f = \bigO(d^2m^d)$, which was to show.
\end{proof}

\section{The subspace of linearizable instances}
\label{sec:subspace}

Let $d\in \nz$, $d\ge 2$, and an acyclic,  $\Pst$-covered  digraph  $G = (V, A)$ with source vertex $s$ and sink vertex $t$  be  fixed. 
Let $H^{(d)} := \set{B \subseteq H \mid |B| \leq d}$ be the set of all subsets of at most $d$ arcs in arc set $H\subseteq A$.
Every order-$d$ cost function $q_d\colon A^{(d)}\to \rz$ can be uniquely represented by a vector $x \in \R^{A^{(d)}}$ with $q_d(F)=x_F$ for all $F\in A^{(d)}$, and vice-versa. Thus, each instance $(G,s,t,q_d)$ can be identified with the corresponding vector $x\in \rz^{A^{(d)}}$ and we will say that $x\in \R^{A^{(d)}}$ is an instance of the SPP$_d$.  In this context, an instance $x\in \R^{A^{(d)}}$ of the SPP$_d$ is linearizable iff there exist an $\bar{x} =(x_a)_{a\in A} \in \R^{A^{(1)}}$ such that $\sum_{S\subseteq P\colon |S|\le d} x_S=\sum_{a\in P}\bar{x}_a$. (Note that sometimes we will slightly abuse the notation and use $A$ instead of $A^{(1)}$.)
Thus, clearly,  that if $x,y \in \R^{A^{(d)}}$ are linearizable instances of the SPP$_d$, then $\mu x + \nu y$ is also a linearizable instance, for all scalars $\mu,\nu \in \R$.
Therefore, the set of linearizable instances of the SPP$_d$ on the fixed digraph  $G$ forms a linear subspace ${\cal L}_d$ of $\R^{A^{(d)}}$. 

Methods to compute this subspace 
are useful in
B\&B algorithms 
for the SPP$_d$ as they can be applied to compute
better lower bounds along the lines of what Hu and Sotirov~\cite{huSo2021} did for general quadratic binary programs.
Hu and Sotirov showed that for $d = 2$ a basis of ${\cal L}_d$ can be computed in polynomial time \cite[Prop. 5]{huSo2021}.
We extend their result 
to the case  $d > 2$.
\begin{theorem}
\label{thm:subspaces}
Let $G = (V, A)$ be  an acyclic, $\Pst$-covered digraph with source vertex $s$ and sink vertex $t$ and let   $d \in \N$ be a constant.  A basis of the subspace ${\cal L}_d$  of the  linearizable instances of the SPP$_d$ on $G$  can be computed in polynomial time.  
\end{theorem}
%
The goal of this section is to prove \cref{thm:subspaces}.  The main idea of the proof relies in specifying   an integer  $k\in \nz$  bounded by  a polynomial  on the size of the input of the SSP$_d$ on $G=(V,A)$ and  a  matrix $M\in \R^{k\times |A^{(d)}|}$ 
such that for  $f: \R^{A^{(d)}} \rightarrow \R^k$ with  $f(x)=Mx$,  we have:
$f(x) = 0$ 
iff
$x$ is  a linearizable instance of the SPP$_d$ on $G$. Thus,  the linearizable instances $x$ of the SPP$_d$ on $G$
form the kernel of $M$, which can be efficiently computed.
If there are no contextual ambiguities we briefly say that the integer $k$ as above is \emph{polynomially bounded}.

The function $f$ will be composed of smaller building blocks, mimicking the way that our algorithm from \cref{subsection:alg} reduces the SPP$_d$ to smaller instances of the APECP$_{d-1}$, which in turn are reduced to instances of the SPP$_{d-1}$. Formally, for a vertex $u$ let $V_u$ ($A_u$) be the set of vertices (arcs)  lying on at least one $s$-$u$-path, as specified in \cref{def:corresponding-instance}. An instance of the SPP$_{d}$ on the smaller digraph $(V_u, A_u)$ with source $s$ and sink $t$ can be interpreted as a vector $x \in \R^{A_u^{(d)}}$. Likewise, an instance of the APECP$_{d}$ on the smaller digraph $(V_u, A_u)$ can be interpreted as a vector $x \in \R^{A_u^{(d)}}$. 
In the following, we show that for all vertices $u \in V$ and every $d \in \N, d\geq 2$, there exists some polynomially bounded $k \in \N$ and a linear function
\[
f_{u,d} : \R^{A_u^{(d)}} \rightarrow \R^k
\]
such that $f_{u,d}(x) = 0$ if and only if $x$ is a YES-instance of the \textsc{lin}SPP$_d$ on the digraph $(V_u, A_u)$. Likewise, we show that for all vertices $u \in V$ and every $d \in \N, d\geq 1$, there exists some polynomially bounded $k \in \N$ and a linear function
\[
g_{u,d} : \R^{A_u^{(d)}} \rightarrow \R^k
\]
such that $g_{u,d}(x) = 0$ if and only if $x$ is a YES-instance of the APECP$_{d}$ on the digraph $(V_u, A_u)$. 

We prove the two claims stated above  by induction. The base case of the induction is concerned with the APECP$_1$. For the remainder of this section, we consider a fixed acyclic and  $\Pst$-covered digraph $G = (V, A)$ with source vertex $s$ and sink vertex $t$ with $n$ vertices and $m$ arcs.

\begin{lemma}
\label{lemma:subspaces-apec-1}
     Consider the APECP$_1$ on the digraph $(V_u, A_u)$ with source $s$ and sink $u$  for some vertex $u\in V$. The following statements  hold:
    \begin{enumerate}[(i)]
        \item There exists $k \in \N$ and a linear function $g_{u,1} : \R^{A_u^{(1)}} \rightarrow \R^k$, such that $k = \bigO(m)$ and such that $g_{u,1}(x) = 0$ iff $x$ is a YES-instance of the APECP$_1$.
        \item There exists a linear function $g'_{u,1} : \R^{A_u^{(1)}} \rightarrow \R$, such that for all $x$ with $g_{u,1}(x) = 0$, we have that $g'_{u,1}(x)$ is the common cost of all $s$-$u$-paths in the APECP$_1$ instance $x$.
        \item  The functions $g_{u,1}, g'_{u,1}$ can be computed in $\bigO(m)$  time.
    \end{enumerate}
\end{lemma}
\begin{proof}
    Assume we are given an instance $x \in \R^{A_u^{(1)}}$. The vector $x$ represents a function $x : A_u^{(1)} \rightarrow \R$. Sometimes we use the equivalent notation   $x(F)$ instead of $x_F$,  for $F\in A_u^{(1)}$. 
    We consider the same dynamic program as in \cref{lemma:apec-1}. For every vertex $w \in V_u$, we fix some arbitrary $s$-$w$-path $P_w$.
    We introduce an auxiliary  variable $y_w$ for every vertex $w \in V_u$ by making use of $P_w$:
    \begin{equation} 
    \label{eq:choosepath} y_w := \sum_{a \in P_w}x(\set{a})\, .\end{equation}
    Clearly,  $y_w$ depends  linearly on $x$. By the same argumentation as in \cref{lemma:apec-1}, the dynamic program correctly concludes that the APECP$_1$ instance is a YES-instance iff
    \[\forall a = (w,z) \in A_u : y_z = y_w + x(\set{a}). \]
    Further, if $x$ is  a YES-instance, then the common cost of all $s$-$u$-paths  equals $y_u + x(\emptyset)$. (Note that  the case $x(\emptyset) \neq 0$ can appear.)

    Now we construct the functions $g_{u,1}, g'_{u,1}$ in the following way. $g_{u,1}$ maps to $\R^k$ with $k = |A_u|$ and is given by
    \[ g_{u,1}(x) = ( y_w + x(\set{a}) - y_z)_{a = (w,z) \in A_u}\, ,\mbox{ for all $x\in \R^{A_u^{(1)}}$} .\]
    The real-valued function $g'_{u,1}$ is given by $g'_{u,1}(x) = y_u + x(\emptyset)$, for all $x\in \R^{A_u^{(1)}}$.
    
    Since $y_w$ depends  linearly on $x$, for any $w\in V_u$, both  functions  $g_{u,1}, g'_{u,1}$  are linear. The  properties stated in  (i) and (ii) are fulfilled  by construction. Further, it is straightforward to see that given an input $x \in \R^{A_u^{(1)}}$, the vector $g_{u,1}(x)$ can be computed in $\bigO(m^2)$ time and the value $g'_{u,1}(x)$ can be computed in $\bigO(m)$ time. Finally, note that $g_{u,1}(x)$ can also be computed  in $\bigO(m)$ time, if  the paths $P_w$ in \cref{eq:choosepath} are chosen in a specific way, namely such that  they    form an out-tree rooted at $s$.  
\end{proof}
%
\begin{lemma}
\label{lemma:subspaces-induction}
    Let $d \geq 2$. Consider the SPP$_d$ and the APECP$_{d}$ on the digraph $(V_u, A_u)$ with source $s$ and sink $u$  for some vertex $u\in V$. The following statements  hold:
    \begin{enumerate}[(i)]
        \item There exists $k \in \N$, $k = \bigO(m^{d})$,  and a linear function $f_{u,d} : \R^{A_u^{(d)}} \rightarrow \R^k$, such that  $f_{u,d}(x) = 0$ iff $x$ is a linearizable instance of the SPP$_{d}$.
        \item There exists a linear function $f'_{u,d} : \R^{A_u^{(d)}} \rightarrow \R^{A_u}$, such that for all $x$ with $f_{u,d}(x) = 0$,   $f'_{u,d}(x) \in \R^{A_u}$ is a linearizing cost function in reduced form for the SPP$_d$ instance $x$.
        \item There exists $k \in \N$, $k = \bigO(m^{d})$,  and a linear function $g_{u,d} : \R^{A_u^{(d)}} \rightarrow \R^k$, such that   $g_{u,d}(x) = 0$ iff $x$ is a YES-instance of the APECP$_{d}$.
        \item There exists a linear function $g'_{u,d} : \R^{A_u^{(d)}} \rightarrow \R$, such that for all $x$ with $g_{u,d}(x) = 0$, $g'_{u,d}(x)$ is the common cost of all $s$-$u$-paths in the APECP$_{d}$ instance $x$.
        \item The functions $f_{u,d}, f'_{u,d}, g_{u,d}, g'_{u,d}$ can be computed in $\bigO(m^d)$ time.
    \end{enumerate}
\end{lemma}
\begin{proof}
    We prove the lemma by induction. \cref{lemma:subspaces-apec-1} shows that the statements in (iii) and (iv) hold  for $d = 1$. For the inductive step with $d\ge 2$  assume that the statements in (iii) and (iv) hold for the  value $d - 1$. Thus, the functions $g_{u,d-1}$ and $g'_{u,d-1}$ fulfilling the properties in (iii) and (iv) exist. We now proceed to show that the statements in (i) -- (iv) hold for the value $d$, in particular, we construct the functions $f_{u,d}, f'_{u,d}, g_{u,d}, g'_{u,d}$ with the corresponding properties.

    Consider the order-$d$ cost function $q_d \in \R^{A_u^{(d)}}$ of  an SPP$_d$ instance (APECP$_d$ instance)  on the graph $(V_u, A_u)$ with source $s$ and sink $u$. For a fixed vertex $u$ such instances of the SPP$_d$ (APECP$_d$) are fully determined by $q_d$. Hence, unless other specified,  we will identify SPP$_d$ instances (APECP$_d$ instances)  as above with the corresponding order-$d$ cost function $q_d$.  Fix some system $N \subseteq A_u$ of nonbasic arcs with respect to $(V_u, A_u)$. Let $F \subseteq A_u$ be the set of strongly basic arcs with respect to $N$, and let $F'$ be the set of arcs which are incident to the source (hence $A_u = N \dot\cup F \dot\cup F'$ is a partition by the definition of basic/strongly basic arcs).  \cref{lemma:property-pi,lemma:corresponding-instance} imply that  $q_d$ is a YES-instance of \textsc{Lin}SPP$_d$ iff   $q^{(a)}_{d-1}$ is a YES-instance of APECP$_{d-1}$, for every arc $a = (w,z) \in F$. By the inductive assumption, the latter is the case iff  $g_{w,d-1}(q^{(a)}_{d-1}) = 0$. Observe that the function which maps $q_d \in \R^{A_u^{(d)}}$ to $q^{(a)}_{d-1} \in \R^{A_w^{(d-1)}}$, $a\in F$,  is a linear function (this follows from the definition, i.e.\ \cref{eq:definition_q_e}).  Therefore, the composition  $f_{u,d}$ defined by 
    \begin{align*}
        f_{u,d}(q_d) = (g_{w,d-1}(q^{(a)}_{d-1}))_{a = (w,z) \in F}, \\
    \end{align*}
     is linear.  The inductive assumption implies  that   $g_{w,d-1}(q^{(a)}_{d-1})\in \rz^{k'}$ with $k'=\bigO(m^{d-1})$, for each $a=(w,z)\in F$. Since   $|F|=\bigO(m)$, the composition $f_{u,d}$ maps to $\R^k$ with $k = \bigO(m k') = \bigO(m^d)$. 
     %By the arguments above,
      Since the instance $q_d$ of the SPP$_d$ is linearizable iff $g_{w,d-1}(q^{(a)}_{d-1}) = 0$, for all $a\in F$, property (i) is satisfied. Further,  we define $f'_{u,d}\colon \R^{A_u^{(d)}}\to \R^{A_u}$ by setting $f'_{u,d}(q_d)=(x_a)_{a\in A_u}\in \R^{A_u}$ with 
    \begin{align*}
        x_a = \begin{cases}
            g'_{w,d-1}(q^{(a)}_{d-1}) & a = (w,z) \in F\\
            0 & a \in N\\
            \text{SPP}_d(a \cdot N_z, q_d) & a = (w,z) \in F'\, .
        \end{cases} \\
    \end{align*}
    Then $f'_{u,d}(q_d)$ is a mapping of $A_u$ to $\R$ and it is a linearizing cost function for the $q_d$ instance of the SPP$_d$ whenever $f_{u,d}(q_d) = 0$. Moreover,  this linearizing function is in reduced form. These facts  follows from the arguments in \cref{lemma:property-pi,lemma:corresponding-instance} (see also the description  right after the proof of  \cref{lemma:corresponding-instance})  and the inductive assumption on $g'_{w,d-1}$. 
    %
    Further, observe that  the function $f'_{u,d}$ is linear by construction,  since $q_{d-1}^{(a)}$ and $g'_{w,d-1}$ are linear for all $a=(w,z)\in F$.
    By applying  the speedup technique from \cref{subsection:alg}  the sequence of values $(q^{(a)}_{d-1})_{a \in F}$, as well as the sequence $(\text{SPP}_d(a \cdot N_z; q_d))_{a\in F}$ can be computed in $\bigO(m^d)$ time for any  given $q_d$. 
    Combined with the inductive assumption that both $g_{w,d-1}, g'_{w,d-1}$ can be computed in $\bigO(m^{d-1})$ time for each vertex $w$, we obtain that both $f_{u,d}, f'_{u,d}$ can be computed in $\bigO(m^d)$ time. Hence we have shown (i) and (ii).
\smallskip

    Finally, we show how to construct $g_{u,d}, g'_{u,d}$ while assuming that the the functions $f_{u,d}$, $f'_{u,d}$ described in (i) and (ii) exists and have the corresponding properties. Let $q_d \in \R^{A_u^{(d)}}$ be an instance of the APECP$_d$. It follows by the arguments from \cref{lemma:reduction-APEC} that $q_d$ is a YES-instance with all $s$-$u$-paths having cost $\beta$ iff $q_d$ is a linearizable instance of SPP$_d$. Moreover,  in the linearizable case, the equality $\red(c) = \const_\beta$ holds for the linearizing function $c$. Then,  the inductive assumption and the uniqueness of the reduced form
    (see \cref{lemma:nonbasic-arcs}) imply that $q_d$ is  a YES-instance of the APECP$_d$  iff the following equalities hold 
    \begin{align*}
        &f_{u,d}(q_d) = 0 \\
        \text{ and } &\forall a,a' \in F':  (f'_{u,d}(q_d))(a) = (f'_{u,d}(q_d))(a') \\
        \text{ and } &\forall a \in A_u \setminus F': (f'_{u,d}(q_d))(a) = 0. 
    \end{align*}
     Then, we set $g_{u,d}:=f_{u,d}$ and observe that $g_{u,d} = 0$ iff all three of the above conditions are satisfied, thus iff  $q_d$ is  a YES-instance of the APECP$_d$. 
    Moreover,  $q_{u,d}$ clearly  maps to  $\R^k$ for some $k = \bigO(m^d)$ because $f_{u,d}$ has this property due to the inductive assumption.   Thus $g_{u,d}$ fulfills the properties stated in (iii).  Further, we define $g'_{u,d}(q_d) := (f'_{u,d}(q_d))(a)$, for some arbitrary arc $a \in F'$. Then  the equality  $\red(c) = \const_\beta$ implies that the property stated in (iv) is fulfilled.    Finally, by the inductive assumption, we can compute $f_{u,d}(q_d), f'_{u,d}(q_d)$ in $\bigO(m^d)$ time. It follows that the functions $g_{u,d}(q_d), g'_{u,d}(q_d)$ can be computed in $\bigO(m^d)$ time. Hence we have shown (iii),  (iv) and (v).
\end{proof}

\begin{proof}[Proof of \cref{thm:subspaces}]
    Consider the sink vertex $t$. By \cref{lemma:subspaces-induction}, the function $f_{t,d}$ has the property that some instance $q_d \in \R^{A_u^{(d)}}$ is linearizable iff $f_{t,d}(q_d) = 0$. 
     For the proof of the theorem it is enough to show  how to efficiently compute a matrix representation of  $f_{t,d}$, i.e.\ a matrix $M\in \R^{|A_u^{(d)}|\times k}$ with  $k = \bigO(m^d)$, such that $f_{t,d}(x) = Mx$ for all $x$. Then  $\mathcal{L}_d$ is equal to the kernel of $M$, thus  it can be  computed in polynomial time with respect to $|A_u^{(d)}|$ and $k$. Since $k =\bigO(m^d)$  and $|A_u^{(d)}| =\bigO(m^d)$ this computation time is polynomial in $\bigO(m^d)$, hence also polynomial with respect to the input of the $q_d$ instance of the SPP$_d$. 

    Now we turn to the efficient computation of $M$. 
    Consider the set $e_1,\dots, e_k$ of all standard basis vectors of $\R^{A_u^{(d)}}$. Clearly,  $k = \bigO(m^d)$. By \cref{lemma:subspaces-induction}, we can compute the vector $f_{t,d}(e_j)$ in $\bigO(m^d)$ time for every $j=1,\dots,k$. Basic linear algebra tells us that these vectors constitute the columns of $M$. We can hence compute the matrix $M$ and a basis of its kernel in $\bigO(m^{2d})$ time,  which is  polynomial with respect to the input of the $q_d$ instance of the SPP$_d$. 
\end{proof}