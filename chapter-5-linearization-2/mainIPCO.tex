\documentclass[10pt,a4paper]{scrartcl}
\usepackage[utf8]{inputenc}
\usepackage{amsmath}
\usepackage{amsfonts}
\usepackage{amssymb}
\usepackage{authblk}
\usepackage{verbatim}
\usepackage{thmtools}
\usepackage{amsthm}
\usepackage{thm-restate}
\usepackage{todonotes}
%\declaretheorem[name=Theorem,numberwithin=section]{thm}
%--experiment
\usepackage%[hidelinks]
			{hyperref}

\usepackage[capitalize, noabbrev]{cleveref}
\usepackage[autostyle=true]{csquotes}
\usepackage{graphicx}
\usepackage{color}
\usepackage{enumerate}
\usepackage[ruled, lined, linesnumbered, commentsnumbered, noend]{algorithm2e}

\usepackage{tikz}
\usetikzlibrary{calc}
\usetikzlibrary{arrows,shapes}
\usetikzlibrary{decorations.markings}
\usetikzlibrary{arrows.meta}

% Abstand obere Blattkante zur Kopfzeile ist 2.54cm - 15mm
%\setlength{\topmargin}{-15mm}


% Umgebungen für Definitionen, Sätze, usw.
% Es werden Sätze, Definitionen etc innerhalb einer Section mit
% 1.1, 1.2 etc durchnummeriert, ebenso die Gleichungen mit (1.1), (1.2) ..
\newtheorem{theorem}{Theorem}[section]
\newtheorem{definition}[theorem]{Definition} 
\newtheorem{lemma}[theorem]{Lemma}
\newtheorem{corollary}[theorem]{Corollary}	
\newtheorem{example}[theorem]{Example}
\newtheorem{observation}[theorem]{Observation}	
\newtheorem{conjecture}[theorem]{Conjecture}
\newtheorem{question}[theorem]{Question}
\newcommand{\ren}[1]{{\color{magenta}{#1}}}
\numberwithin{equation}{section}
\renewcommand\Affilfont{\small}
\newtheorem{defi}[theorem]{Definition}

\title{
A fast for linearizing the quadratic shortest path problem and
its generalizations\\
\ren{A fast algorithm for the linearization of the quadratic and higher-order shortest path problem}}
	\author[1]{Eranda \c{C}ela\footnote{cela@math.tugraz.at}}
	\author[1]{Bettina Klinz\footnote{klinz@opt.math.tugraz.at}}
    \author[2]{Stefan Lendl\footnote{stefan.lendl@uni-graz.at}}
    \author[3]{Gerhard Woeginger}
    \author[1]{Lasse Wulf\footnote{wulf@math.tugraz.at}}
	
	\affil[1]{Institute of Discrete Mathematics, Graz University of Technology, Austria}
	\affil[2]{Institut of Operations and Information Systems, University of Graz, Austria}
	\affil[3]{Department of Computer Science, RWTH Aachen, Germany}
	
	\date{}

\newcommand{\R}{\mathbb{R}}
\newcommand{\rz}{\mathbb{R}}
\newcommand{\N}{\mathbb{N}}
\newcommand{\Z}{\mathbb{Z}}
\newcommand{\nz}{\mathbb{Z}}
\newcommand{\set}[1]{\{ #1 \}}
\newcommand{\fromto}[2]{\set{#1, \ldots, #2}}

\newcommand{\bigO}{\mathcal{O}}
\newcommand{\powerset}{\mathcal{P}}
\newcommand{\dotunion}{\mathbin{\dot{\cup}}}
\newcommand{\const}{\textsf{source}}

\newcommand{\True}{\textsc{True}}
\newcommand{\False}{\textsc{False}}
\newcommand{\Pst}{\mathcal{P}_{st}}
\newcommand{\Nscript}{N}

\newcommand{\boxxx}[1]
 {\fbox{\begin{minipage}{11.80cm}\begin{center}\bigskip\begin{minipage}{11.30cm}
  #1\end{minipage}\end{center}~\end{minipage}}}

%\newcommand{\comment}[1]{\textcolor{blue}{(#1)}}


\DeclareMathOperator{\red}{\textsf{reduced}}
\DeclareMathOperator{\val}{val}




%Special short commands for interval interdiction notation


\begin{document}
\maketitle
\begin{abstract} 
We provide an algorithm that solves the linearization problem for the quadratic
shortest path problem on acyclic digraphs.
The running time is linear in the input size, beating the previously best known
algorithm by a factor of $\bigO(nm)$.
The algorithm is based on a new insight, which states that linearizability of
the quadratic shortest path problem for acyclic graphs can be understood as a
local property. \ren{\{}The algorithm also solves the more general degree-$d$ 
linearization problem in running time linear in the input size.\ren{\}}
\ren{The algorithm also solves the 
linearization problem for the more general higher-order shortest path
problem. The running time remains   linear in the  input size.}
\end{abstract}

\noindent\textbf{Keywords:} quadratic shortest path problem; 
higher-order shortest path problem, linearization;

\input{introIPCO}
\section{Notations  and preliminaries}\label{defi:sec}

 Given a digraph $G=(V,A)$, a simple directed $s$-$t$-path $P$ in $G$ is specified as a sequence of
arcs $P=(a_1,a_2,\ldots,a_p)$ such that  $a_1$ starts at $s$, $a_p$
ends at $t$, nonconsecutive arcs do not share a vertex and  the end vertex of $a_{i}$ coincides with  the start vertex of $a_{i+1}$ for any $i\in \{1,\ldots,p-1\}$. The number
$p$ of arcs in $P$ is called the length of the path. We sometimes use the same notation for a path $P$ and the  set of its arcs. 
%Alternatively, a simple directed $s$-$t$-path $P$ of length $p$ in $G$  is specified as a sequence of pairwise disjoint vertices $(u_0,u_1,\dots,u_p)$, 
%where $x_0=s$, $u_p=t$ and $(u_i,u_{i+1})\in A$ for all $i\in\{0,\dots,p\}$. Thus  $a_i=(u_{i-1},u_i)$, for $i\in \{1,\dots,p\}$. 
We  consider a single arc $(x, y)$ as an $x$-$y$-path of length $1$  and a single vertex $x$ as a  trivial $x$-$x$-path of length $0$. Given an  $x$-$y$-path $P_1$ and a  $y$-$z$-path $P_2$, we denote the \emph{concatenation} of $P_1$ and $P_2$ by $P_1 \cdot P_2$. We also consider concatenations of paths and arcs, that is, terms of the form $P \cdot a$ for some $x$-$y$-path $P$ and some arc $a = (y, z)$.
 
In the linearization problem, we are concerned with acyclic digraphs $G =(V, A)$ with a source vertex $s$ and a sink vertex $t$. 
We denote by  $\Pst$ the set of all simple directed $s$-$t$-paths.
We often assume that $G$ is \emph{$\Pst$-covered}, that is, every arc in $G$ is traversed by at least one path in $\Pst$. 
Note that we can make this assumption without loss of generality: If some arc is not traversed by at least one $s$-$t$-path, then it has no effect on the linearizability of the instance, and so we can delete that arc.

Let $d \geq 2$ be a natural number. The \emph{Order-$d$ interaction costs} are given by a mapping $q_d\colon \set{B \subseteq A : |B| \leq d}\to \rz$, assigning a (potentially negative) interaction cost to every subset of at most $d$ arcs.
The cost $\text{SPP}_d(P,q_d)$ of some path $P$ under interaction costs $q_d$ is defined as in equation (\ref{dSPP:obj}).
If $d$ is unambiguously clear form the context, we use the  more compact notation $f_q(P): = \text{SPP}_d(P,q_d)$.
 In this chapter we explicitly allow the case $q(\emptyset) \neq 0$, because this simplifies the calculations.  
The \emph{linearization problem for the Order-$d$ Shortest Path Problem} (\textsc{Lin}SPP$_d$)
is formally defined as follows.
\begin{center}
%%%%%%%%%%%%%%%%%
\boxxx{\textbf{Problem:} The  linearization problem for the SPP$_d$ (\textsc{Lin}SPP$_d$)
\\[1.0ex]
\textbf{Instance:} A $\Pst$-covered directed graph $G=(V,A)$ with $s,t\in V$, $s\neq t$; an integer $d \geq 2$;
an order-$d$ arc interaction cost function $q_d: \set{B \subseteq A \colon |B| \leq d} \to \R$.
\\[1.0ex]
\textbf{Question:} Find a \emph{linearizing cost function}  $c\colon A\to\R$  such that $\text{SPP}_d(P,q_d) = \text{SPP}(P,c)$ for all $P \in \Pst$ or decide that such a linearizing  cost function does not exist.}
%%%%%%%%%%%%%%%%%
\end{center}
%%%%%%%%%%%%
In the special case $d = 2$, we obtain the    linearization problem for the QSPP (\textsc{Lin}QSPP).
\smallskip

Finally, let us consider the \emph{Generic Shortest Path Problem} (GSPP) which takes as input  a digraph $G=(V,A)$ with a source vertex $s$, a sink vertex $t$, $s\neq t$, and a generic cost function $f\colon \Pst \to \rz$  assigning  a cost $f(P)$ to every path $P\in \Pst$\footnote{We assume that $f$ is specified by an oracle.}. We assume w.l.o.g.\ that $G$ is $\Pst$-covered. The goal is to find an $s$-$t$-path  which minimizes the objective function $f(P)$ over $\Pst$. 
A linearizable instance of  the GSPP and the linearization problem for the GSPP (\textsc{Lin}GSPP) are  defined analogously as in the respective definitions for SPP$_d$.



We note that in our \cref{def:linearizable} we allowed the case that the linearizing function $c : A \to \R$ can take negative values. This could be a problem, since if a digraph contains negative cycles, the shortest path problem is NP-hard in general. However, as we are only concerned with acyclic digraphs in this chapter, this is not a problem. In fact, the following lemma shows that for acyclic digraphs it is a more difficult problem to decide if there is a linearizing function $c : A \to \R$ than the problem to decide if there is a linearizing function $c': A \to \R_+$. Hence, we consider only the first problem for the rest of the chapter.
\begin{lemma}
If $(G,s,t,f)$ is an instance of the GSPP such that $G  = (V,A)$ is an acyclic digraph, then there is a nonnegative linearizing function $c' : A \to \R_+$ if and only if there is a linearizing function $c : A \to \R$ and $f(P) \geq 0$ for all $s$-$t$-paths $P$.
\end{lemma}
\begin{proof}
If there is no linearizing function $c : A \to \R$, then in particular there is also no linearizing function $c' : A \to \R_+$. On the other hand, if there is a linearizing function $c : A \to \R$, then we distinguish two cases. If there is an $s$-$t$-path $P$ with $f(P) < 0$, then clearly there cannot exist a linearizing cost function  $c' : A \to \R_+$, because $P$ would receive the incorrect costs $c'(P) \geq 0$. On the other hand, if $f(P) \geq 0$ for all $s$-$t$-paths, we claim that we can find nonnegative $c'$ as desired. Indeed, for all vertices $v \in V$, we let $\pi(v)$ be the cost of a shortest $s$-$v$-path with respect to $c$. Since $G$ has no cycles, in particular no negative cycles we have that for all arcs $(u,v) \in A$ the inequality $\pi(v) \leq \pi(u) + c(u,v)$ holds. We define the adjusted costs $c_\pi(u,v) := \pi(u) - \pi(c) + c(u,v)$ for all arcs $(u,v) \in A$. By the previous inequality we have $c_\pi(u,v) \geq 0$. By a telescope argument we have $c_\pi(P) = c(P) + \pi(s) - \pi(t) = c(P) - \pi(t)$ for all $s$-$t$-paths $P$. Finally, we define $c'$ by letting $c'(a) := c_\pi (a) + \pi(t)$ for all arcs incident to the source and $c'(a) := c_\pi(a)$ otherwise. Note that $\pi(t) \geq 0$ due to the assumption $f(P) \geq 0$ for all $P \in \Pst$. Since $c_\pi$ and $\pi(t)$ are both nonnegative, $c'$ is also nonnegative. By construction we have $c'(P) = f(P)$ for all $s$-$t$-paths $P$, so $c'$ is a nonnegative linearizing function. 
\end{proof}

\section{A characterization of linearizable instances of  the GSPP}\label{charact:sec}
 %%%%%%%%%%%%%%%
 The main result of this section is \cref{thm_linearization_characterization}, our novel characterization of  linearizable instances of the GSPP on acyclic digraphs defined as in Section~\ref{defi:sec}. 
 %%%%%%%%%%%%
\begin{definition}
Let $G=(V,A)$ be a $\Pst$-covered acyclic digraph. For some vertex $v$, let $P_1, P_2$ be two $s$-$v$-paths, and let $Q_1, Q_2$ be two $v$-$t$-paths. The 5-tuple $(v,P_1,P_2,Q_1,Q_2)$ is called a \emph{two-path system} contained in $G$. The system is called \emph{linearizable} with respect to the function $f : \Pst \rightarrow \R$, if there exists a cost function $c : A \rightarrow \R$ such that for all four paths $P \in \set{P_1 \cdot Q_1, P_1 \cdot Q_2, P_2 \cdot Q_1, P_2 \cdot Q_2}$ we have $f(P) = SPP(P,c)$. Such a $c$ is called  a {\emph linearizing cost function} for  $(v,P_1,P_2,Q_1,Q_2)$ with respect to $f$. 
\end{definition}

%--------------------------------------------------------------------
%Create custom tikz style for directed edges with arrow in the middle
\tikzset{->-/.style={
	decoration={
 		markings,
  		mark=at position #1 with {\arrow[scale=2,>=stealth]{>}}
  		},
  	postaction={decorate}
  },
  ->-/.default=0.5
}%--------------------------------------------------------------------
\tikzstyle{vertex}=[draw,circle,fill=black, minimum size=4pt,inner sep=0pt]
\tikzstyle{edge} = [draw,thick,-]
\tikzstyle{weight} = [font=\small]
\begin{figure}[bth]
\centering
\begin{tikzpicture}[scale=1.0, auto,swap]

    \node[vertex] (s) at (0,0) {};
    \node[vertex] (v) at (4,0) {};
    \node[vertex] (t) at (8,0) {};

    \node[above] at (s) {$s$};
    \node[above] at (t) {$t$};
    \node[above] at (v) {$v$};
    
    \draw[->-=0.6,pos=0.4] (s) to[bend left] node[above]{$P_1$} (v);
    \draw[->-=0.6,pos=0.4] (s) to[bend right] node[above]{$P_2$} (v);
    \draw[->-=0.6,pos=0.4] (v) to[bend left] node[above]{$Q_1$} (t);
    \draw[->-=0.6,pos=0.4] (v) to[bend right] node[above]{$Q_2$} (t);
\end{tikzpicture}
\caption{A two-path system.}
 \label{fig:two-path-system}
\end{figure}

See \cref{fig:two-path-system} for an illustration of a two-path system. 
Note that $P_1$ and $P_2$ (as well as $Q_1$ and $Q_2$) can have common inner vertices and that the cases $P_1=P_2$, $Q_1=Q_2$, $v = s$ and $v = t$ are allowed. 
However, due to the acyclicity of $G$, the paths $P_i$ and $Q_j$ have only the vertex $v$ in common for $i,j \in\{1,2\}$. Further, observe that the linearizability of a two-path system 
is a local property, in the sense  that it only   depends on the four paths $P_1 \cdot Q_1, P_1 \cdot Q_2, P_2 \cdot Q_1$ and $P_2 \cdot Q_2$. Indeed,  the following simple characterization holds. 

\begin{proposition}
\label{obs:linearizability-two-paths}
A two-path system $(v,P_1,P_2,Q_1,Q_2)$ is linearizable with respect to some function $f\colon \Pst\to \rz$ iff
\begin{equation}
f(P_1 \cdot Q_1) + f(P_2 \cdot Q_2) = f(P_1 \cdot Q_2) + f(P_2 \cdot Q_1).    \label{eq:two-path-lin}
\end{equation}
\end{proposition}
\begin{proof}
    First, assume that $(v,P_1,P_2,Q_1,Q_2)$ is linearizable and let $c$ be the corresponding  linearizing cost function.  Let $M_1$ ($M_2$) be the multiset  resulting from   the union of the sets of the arcs of the paths $P_1\cdot Q_1$ and $P_2\cdot Q_2$  ($P_1\cdot Q_2$ and $P_2 \cdot Q_1$). Since $M_1$ and $M_2$ coincide we get   $c(P_1 \cdot Q_1) + c(P_2 \cdot Q_2) =\sum_{a\in M_1} c(a)= \sum_{a\in M_2} c(a)=c(P_1 \cdot Q_2) + c(P_2 \cdot Q_1)$. Then,   (\ref{eq:two-path-lin}) follows from the definition of the linearizability of   $(v,P_1,P_2,Q_1,Q_2)$.
    
    Assume now  that \cref{eq:two-path-lin} is true. We  show the linearizability of  the two-path system with respect to $f$ by constructing a linearizing cost function $c$.   It is easy to find a suitable  $c$ if $P_1=P_2$ or $Q_1=Q_2$. So let us consider  the more general case  where $P_1\neq P_2$ and $Q_1\neq Q_2$.  In this case, for   each  $P\in \{P_1,P_2,Q_1,Q_2\}$ there exists a (not necessarily unique) \emph{representative arc} $a\in P$ such that $a$  is not contained in any other path  $Q\in \{P_1,P_2,Q_1,Q_2\}$, $Q \neq P$. Let $a_1$, $a_2$, $e_1$, $e_2$ be the representative arcs of $P_1$, $P_2$, $Q_1$ and $Q_2$, respectively.  Consider now a cost function $c : A \rightarrow \R$, such that $c(a) = 0$ if  $a\not\in \{a_1, a_2, e_1, e_2\}$, and $c(a_1)$, $c(a_2)$, $c(e_1)$, 
$c(e_2)$ fulfill the following linear equations:
    \begin{equation*}
        \begin{array}{llcllcl}
        c(a_1) & & + & c(e_1) & & = f(P_1Q_1) \\
        c(a_1) & & + & & c(e_2) &= f(P_1Q_2) \\
        &c(a_2) & + & c(e_1) & &= f(P_2Q_1) \\
        &c(a_2) & + & & c(e_2) &= f(P_2Q_2) 
    \end{array}
    \end{equation*}
    Using basic linear algebra, one can see that this system indeed has a solution whenever  \cref{eq:two-path-lin} holds (there is even a solution with $c(e_2) = 0$). Thus, $c$ constructed as above is a linearizing  cost function for $(v,P_1,P_2,Q_1,Q_2)$ with respect to $f$. 
    \qed
\end{proof}

Now, consider an instance of the GSPP with a $\Pst$-covered acyclic digraph $G$, with a  source vertex $s$, a sink vertex $t$ and a generic cost function $f\colon \Pst\to \rz$. When is this instance $(G,s,t,f)$ linearizable? Obviously, if $G$ contains a two-path system which is not linearizable with respect to $f$, then the instance $(G,s,t,f)$ as a whole is also not linearizable. Interestingly, this necessary condition turns out to be also sufficient.

\begin{theorem}
\label{thm_linearization_characterization}
Let $G$ be a $\mathcal{P}_{st}$-covered acyclic digraph with a source vertex $s$ and a sink vertex $t$ and  let $f : \Pst \rightarrow \R$ be a generic cost function. Then the instance $(G,s,t,f)$ of the GSPP  is linearizable if and only if every two-path system contained in $G$ is linearizable with respect to $f$.
\end{theorem}

 Before proving the theorem, we need some preparation. Let $G = (V,A)$ be a $\Pst$-covered acyclic digraph with source vertex $s$ and sink vertex $t$. First we introduce a \emph{topological arc order} as a total 
  order $\preceq$ on $A$ such that for any pair of  arcs $a$, $a'$ in $A$ the following holds:   if there  exists a path $P$  containing both $a$ and $a'$ such that $a$ comes before $a'$ in $P$, then $a\preceq a'$.  It is easy to see  that any acyclic digraph has a (in general non-unique)  topological arc order. Moreover, a topological arc order  can be obtained from a topological vertex order. 
 
 Further, we recall the definition of a \emph{system of nonbasic arcs}  introduced by  Sotirov and Hu~\cite{huSo2021}.  
 \begin{definition}\label{nonbasic:def}
 Let $G$ be a $\mathcal{P}_{st}$-covered acyclic digraph with a source vertex $s$ and a sink vertex $t$. A set $\Nscript \subseteq A$ is called a \emph{system of nonbasic arcs}, iff for every vertex $v \in V \setminus \set{s,t}$ exactly one of the arcs starting at $v$ is contained in $\Nscript$. The latter  arc is called the \emph{nonbasic arc of $v$}. An arc $a \in A \setminus N$ is called \emph{basic}.
 \end{definition}
 Obviously, the system of nonbasic arcs is not unique.  Any such system  forms an in-tree rooted at $t$ containing   all the vertices in $V$ except for $s$. For some system of nonbasic arcs $\Nscript$ and some vertex $v \in V \setminus \set{s}$, we let $N_v$ denote the unique $v$-$t$-path consisting  of nonbasic arcs (where $N_t$ is the trivial path).  A  cost function $c\colon A \rightarrow \R$ is called \emph{in reduced form} with respect to $\Nscript$, if $c(a) = 0$ for all nonbasic arcs $a \in \Nscript$. The following lemma is an easy adaption from \cite{huSo2021}, where an analogous statement was proven for the less general case of the QSPP instead of the GSPP.


\begin{lemma}[adapted from {\cite[Prop. 4]{huSo2021}}]
\label{lemma:nonbasic-arcs}
Let $G$ be a $\Pst$-covered acyclic digraph with a source vertex $s$ and a sink vertex $t$. Let  $f\colon  \Pst \rightarrow \R$ be a generic cost function and  let $\Nscript \subseteq A$ be a fixed system of nonbasic arcs. If  $(G,s,t,f)$ is a linearizable instance of the GSPP, then there exists one and only one linear cost function $c\colon A \rightarrow \R$ which is both a linearizing cost function  and in reduced form.
\end{lemma}
\begin{proof}
    We have to prove both existence and uniqueness. For the existence, by assumption we have that $(G,s,t,f)$ is a linearizable instance. Hence there exists a linearizing function $c : A \rightarrow \R$, not necessarily in reduced form. Consider some vertex $v \in V \setminus \set{s,t}$ and its nonbasic arc $a_v$. Consider the following modification of the function $c$: Let $\beta = c(a_v)$, then reduce the cost of each outgoing arc of $v$ by $\beta$, and increase the cost of each incoming arc of $v$ by $\beta$. This operation sets the cost of $a_v$ to $0$ and does not change the linear cost of any $s$-$t$-path. Now let $v_1,\dots,v_n$ be a topological vertex order with $v_1 = s$ and $v_n = t$. We repeat the described operation for every vertex $v_{n-1}, v_{n-2}, \dots, v_2$  in this order. It is easily verified that the obtained cost function is a linearization of $(G, f)$ and is in reduced form.
    
    For the uniqueness, assume that there are two distinct linearizing functions $c, c' : A \rightarrow \R$ with the property that all nonbasic arcs have value 0. Consider some topological arc order $\preceq$ and let $a = (u,v)$ be the first arc in the order such that $c(a) \neq c'(a)$. There exists an $s$-$u$-path $P$, because $G$ is $\Pst$-covered. The path $R := P \cdot a \cdot N_v$ is an $s$-$t$-path. By assumption, we have $c(P) = c'(P)$ and $c(N_v) = c'(N_v) = 0$. But then $c(R) \neq c'(R)$, a contradiction. 
    \qed
\end{proof}


Let $(G,s,t,f)$ be a linearizable  instance of the GSPP with $G=(V,A)$ and $\Nscript\subseteq A$ be a fixed system of nonbasic arcs. For a  linearizing cost function $c \colon A \rightarrow \R$, we denote by $\red(c)$ the unique linearizing cost function  in reduced form (which exists due to \cref{lemma:nonbasic-arcs}). 
It follows from the arguments in the proof of  \cref{lemma:nonbasic-arcs} that for given $c$ one can compute $\red(c)$ in $\bigO(n +m)$ time. We are now ready to prove our main theorem.
\begin{proof}[Proof of \cref{thm_linearization_characterization}]
The necessity of the conditions for linearizability is trivial.
Now we prove the sufficiency.
Thus we  assume that every two-path system is linearizable with respect to $f$ and show that $(G,s,t,f)$ is linearizable. Let $\Nscript$ be a system of nonbasic arcs. 
    We consider a topological arc order $\preceq$ on the set  $A$ of arcs in $G$ and inductively define a linearizing cost function $c \colon A \rightarrow \R$ as follows. 
    For any arc $a = (u,v)$ consider some arbitrary $s$-$u$-path and set
        \begin{equation}\label{lincost:equ} c(a) := \begin{cases}
        f(P \cdot a \cdot N_v) - \sum_{a' \in P}c(a'); 
        & a \not\in \Nscript \\
        0%
        ; & a \in \Nscript
        \end{cases}
        \end{equation}
   \smallskip

   The main idea behind this definition is the following: Due to \cref{lemma:nonbasic-arcs}, whenever we look for a linearizing function, we can w.l.o.g. look for one in reduced form. So imagine we have a linearizing function $c'$ such that already $c'(a) =  0$ for all nonbasic arcs. It is not hard to see that \cref{lincost:equ} is a necessary condition on $c'$ that must be true for every $s$-$u$-path $P$ (since all arcs after $a$ on the path $P \cdot a \cdot N_v$ have cost 0). This gives us an initial idea to define $c$. Now consider the following claim.
   
   \textbf{Claim:} If all two-path systems in $G$ are linearizable with respect to $f$, then 
   \begin{enumerate}[(i)]
       \item Function $c$ in Equation~(\ref{lincost:equ}) is well-defined and independent of the concrete choice of $P$.
       \item The following equation holds for all arcs $(u,v) \in A$ and all $s$-$u$-paths $P$:
    \begin{equation}\label{claim2:equ} f(P \cdot a \cdot N_v)=c(a)+\sum_{a' \in P} c(a') = c(P \cdot a \cdot N_v)\end{equation}
   \end{enumerate}
    
     Observe that  the claim immediately implies  that $(G,s,t, f)$ is linearizable. Indeed, let $c$ be the  cost function defined in \cref{lincost:equ} and  let $Q$ be some $s$-$t$-path. Choose $a = (x,t)$ to be the  last arc on $Q$. Then $N_t$ is the trivial path from $t$ to $t$, so by  applying \cref{claim2:equ} to the arc $a$, we have $f(Q) = c(Q)$. 

%%%%%%%%%%%%%%%%%%%%%%%%%%%%%%%%%
%%%%%%%%%%%%%%%%%%%%%%%%%%%%%%%%%
\begin{figure}[bth]
\centering
\begin{tikzpicture}[scale=1.0, auto,swap]
    \node[vertex] (s) at (0,0) {};
    \node[vertex] (u) at (4,0) {};
    \node[vertex] (t) at (9,0) {};
    \node[vertex] (uQ) at (3,0.5) {};
    \node[vertex] (uP) at (3,-0.5) {};
    \node[vertex] (v) at (5.3,0) {};

    \node[above] at (s) {$s$};
    \node[above] at (t) {$t$};
    \node[above] at (u) {$u$};
    \node[above] at (v) {$v$};
    \node[above] at (uQ) {$u_Q$};
    \node[above] at (uP) {$u_P$};
    
    \draw[->-=0.6,pos=0.4,dashed] (s) to[bend right] node[above]{$P$} (uP);
    \draw[->-=0.6,pos=0.4,dashed] (s) to[bend left] node[above]{$Q$} (uQ);
    \draw[->-=0.6,pos=0.4,dashed] (u) to[bend left] node[above]{$N_u$} (t);
    \draw[->-=0.6,pos=0.4,dashed] (v) to node[above]{$N_v$} (t);
    \draw[->-=0.6,pos=0.4] (u) to node[below]{$a$} (v);
    \draw[->-=0.7] (uP) to (u);
    \draw[->-=0.7] (uQ) to (u);
    
\end{tikzpicture}
\caption{Situation in the proof of \cref{thm_linearization_characterization}. The dashed lines represent paths. The arc $(u,v)$ and the two-path system $(u,P,Q,N_u, a\cdot N_v)$ play a vital role.}
 \label{fig:specific-system}
\end{figure}
%%%%%%%%%%%%%%%%%%%%%%%%%%%%%%%%%
%%%%%%%%%%%%%%%%%%%%%%%%%%%%%%%%%
    \textit{Proof of the claim.} We use induction over $\preceq$. For each arc $a = (u, v)$ in $A$, we distinguish between three cases. A sketch of the situation is provided in \cref{fig:specific-system}.
    
    \textbf{Case 1: $u = s$.} This is the base case of the induction. If $a$ is incident to the source vertex, then item (i) holds, because the only $s$-$u$-path is the trivial path. Item (ii) holds by the definition of $c(a)$, and because all nonbasic arcs $a'$ have $c(a') = 0$.
    
    \textbf{Case 2: $u \neq s$ and $a \not\in N$.} Let $a$ be basic and not incident to the source. By the induction hypothesis, $c(a')$ is well-defined for all arcs $a'$ preceding $a$. Hence for the proof of item (i), it remains to show that $c(a)$ is independent of the choice of $P$. Let $Q$ be a second $s$-$u$-path besides $P$, we have to show that 
    \[
    f(P \cdot a \cdot N_v) - \sum_{a' \in P}c(a') = f(Q \cdot a \cdot N_v) - \sum_{a' \in Q}c(a').
    \]
    To see this, let $(u_P, u)$ be the last arc on the path $P$, and let $(u_Q, u)$ be the last arc on the path $Q$ (we use here that $u \neq s$). By the induction hypothesis (ii) applied to $(u_P, u)$, we have that $f(P \cdot N_u) = c(P \cdot N_u) = c(P)$, analogously we have $f(Q \cdot N_u) = c(Q \cdot N_u) = c(Q)$. Furthermore, because the two-path system $(u, P, Q, N_u, a \cdot N_v)$ is linearizable, we have by \cref{obs:linearizability-two-paths}, that $f(P \cdot N_u) + f(Q \cdot a \cdot N_v) = f(P \cdot a \cdot N_v) + f(Q \cdot N_u)$. Putting everything together, we have
    \begin{align*}
         f(P \cdot a \cdot N_v) - c(P) &= f(Q \cdot a \cdot N_v) + f(P \cdot N_u) - f(Q \cdot N_u) - c(P)\\
         &= f(Q \cdot a \cdot N_v) + c(P) - c(Q) - c(P)\\
         &= f(Q \cdot a \cdot N_v) - c(Q),
    \end{align*}
    which was to show. This proves item (i). Item (ii) immediately follows from (i), the definition of $c(a)$ and the fact that all nonbasic arcs $a'$ have cost $c(a') = 0$.
    
    \textbf{Case 3: $u \neq s$ and $a \in N$.} Finally, if $e$ is nonbasic, then (i) is trivial. Furthermore, let $(u_P, u)$ be the last arc on the path $P$ and let $P'$ be the subpath of $P$ without the last arc. Because $a \in \Nscript$, the two paths $P' \cdot (u_P, u) \cdot N_u$ and $P \cdot a \cdot N_v$ are equal, so (ii) follows by induction applied to the arc $(u_P, u)$.
    \qed
\end{proof}

Since in general a graph contains exponentially many different two-path systems,   \cref{thm_linearization_characterization} does not seem to lead to  an  efficient algorithm for  the linearization problem \textsc{Lin}GSPP at a first glance. However, we show in the next section that this is indeed the case. The arguments  are based on a more technical version of \cref{thm_linearization_characterization} and involve the concept of  so-called \emph{strongly basic arcs} and their property $(\pi)$ defined below.

\begin{definition}\label{stronglybasic:def}
    Let $G = (V, A)$ be an acyclic $\Pst$-covered digraph with source vertex $s$ and sink vertex $t$.  Let $f \colon  \Pst \rightarrow \R$ be a generic cost function and let $N \subseteq A$ be a system of nonbasic arcs in $G$. A basic arc $(u, v)$ is called \emph{strongly basic}, if it is not incident to the source vertex, that is if $u \neq s$.
    
  \noindent  A strongly basic arc $a = (u,v)$ has the \emph{property $(\pi)$}, if for 
    any $s$-$u$-paths $P$ the value $ \val(a, P) :=  f(P \cdot a \cdot N_v) - f(P \cdot N_u)$ does not depend  on the choice of $P$. 
    \end{definition}
    
    Thus, if a strongly basic arc $a = (u,v)$ has the property $(\pi)$,   we have  $\val(a, P) = \val(a, Q)$ for any two  $s$-$u$-paths $P, Q$ and this implies the existence of a value   $\val(a) := \val(a, P)$ for each  $s$-$u$-path $P$ and $\val(a)$ is well defined for each strongly basic arc. Finally, note that by definition, the arc set $A$ is partitioned into the three disjoint sets of strongly basic arcs, nonbasic arcs, and arcs incident to $s$. 


\begin{lemma}
\label{lemma:property-pi}
    Let $G = (V, A)$ be an acyclic $\Pst$-covered digraph with source vertex $s$ and sink vertex $t$. Let $f \colon \Pst \rightarrow \R$ be a generic cost function and let $N \subseteq A$ be a system of nonbasic arcs in $G$. Then $(G, s,t,f)$ is linearizable if and only if every strongly basic arc has the property $(\pi)$. In this case, the mapping   $c \colon  A \rightarrow \R$ given by
    \[
    c(a) = \begin{cases}
    \xval(a); & a \text{ is strongly basic }\\
    f(a \cdot N_v); & a = (s,v) \text{ is incident to }s\\
    0; & a \text{ is nonbasic}
    \end{cases}
    \]
    is a  linearizing cost function  in reduced form.
\end{lemma}

\begin{proof}
    Let $a = (u, v)$ be a strongly basic arc. We claim that $a$ has the property $(\pi)$ iff for any two  $s$-$u$-paths $P$, $Q$  the two-path system $(u,P,Q,N_u,a \cdot N_v)$ is linearizable with respect to $f$. Indeed, note that by \cref{obs:linearizability-two-paths}, the   two-path system above is linearizable with respect to $f$  iff $f(P \cdot a \cdot N_v) + f(Q \cdot N_u) = f(P \cdot N_u) + f(Q \cdot a \cdot N_v)$. The latter equation is equivalent to $\xval(a,Q) = \xval(a,P)$. Recalling that  the latter equality  
     holds for every pair of $P, Q$ iff $a$ has the property $(\pi)$ completes the proof of the claim.
    
    Now, assume that some strongly basic arc $(u,v)$ does not have the property $(\pi)$. Then, the corresponding two-path system $(u,P,Q,N_u,a \cdot N_v)$ is not linearizable with respect to $f$ and therefore,  $(G,s,t,f)$ is  not linearizable.
    
    Finally, assume that every strongly basic arc has the property $(\pi)$. 
  In the proof of \cref{thm_linearization_characterization}
    we  use the linearizability assumption   only for  specific two-path systems  of the  form $(u, P, Q, N_u, a \cdot N_v)$,  where   $a = (u, v)$ is some strongly basic arc. Thus,  if the property $(\pi)$ holds for all strongly basic arcs, then  each such specific two-path system is linearizable with respect to $f$ and the linearizability of $(G,s,t,f)$ follows. Furthermore, the value $c(a)$ of the linearizing cost function in \cref{lincost:equ}   equals  $\xval(a)$ for  any arc $a$ which is  strongly basic, equals $f(a \cdot N_v)$ for any arc $(s,v)$ incident to $s$, and equals $0$ for any   nonbasic arc $a$. 
    %This completes the proof of the lemma.
    \qed
\end{proof}


\section{A linear time algorithm for the \textsc{Lin}SPP$_d$}\label{algo:sec}
%%%%%%%%%%%%%%%%%
In this section, we describe an algorithm which solves the linearization problem for SPP$_d$ (\textsc{Lin}SPP$_d$) in  $\bigO(m^d)$ time, i.e., in linear time.
The algorithm uses the relationship between the \textsc{Lin}SPP$_d$ and the  \emph{All-Paths-Equal-Cost Problem (APECP)} which we introduce in  Section~\ref{APEC:ssec}.  
In \cref{subsection:alg} we describe the SPP$_d$ algorithm and discuss
its running time.

 \subsection{The All Paths Equal Cost Problem of Order-$d$ (APECP$_d$)}
\label{APEC:ssec}

The All Paths  Equal Cost Problem of Order-$d$ (APECP$_d$) is defined as follows.
\begin{center}
%%%%%%%%%%%%%%%%%
\boxxx{\textbf{Problem:}  ALL PATHS EQUAL COST of Order-$d$ (APECP$_d$)
\\[1.0ex]
\textbf{Instance:} An acyclic $\Pst$-covered directed graph $G=(V,A)$ with a source vertex $s$ and a sink vertex $t$, an  integer $d \geq 1$;
an order-$d$   cost function $q_d\colon  \set{B \subseteq A : |B| \leq d} \to \R$.
\\[1.0ex]
\textbf{Question:} Do all $s$-$t$-paths  have the same cost, i.e.\ is there some $\beta \in \R$ such that $\text{SPP}_d(P,q_d) = \beta$ for every path $P$ in $\Pst$?}
%%%%%%%%%%%%%%%%%
\end{center}

In the following we establish a connection between the \textsc{Lin}SPP$_d$ and the APECP$_{d-1}$ for $d\ge 2$. More precisely, we show in Lemma~\ref{lemma:corresponding-instance}  that    an   instance $(G,s,t,q_d)$ of the  \textsc{Lin}SPP$_d$ with an acyclic  $\Pst$-covered digraph $G=(V,A)$ can  be reduced  to $\bigO(m)$ instances of APECP$_{d-1}$, each of them corresponding to exactly one strongly basic arc with respect to some fixed system of nonbasic arcs (see Definitions~\ref{nonbasic:def} and \ref{stronglybasic:def}).  The  APECP$_{d-1}$ instance corresponding to a strongly basic arc  $a = (u, v)$ is defined as follows.
\begin{definition}
\label{def:corresponding-instance}
    %Let an instance of the degree-$d$ linearization problem with $d \geq 2$ be given by a $\Pst$-covered acyclic graph $G = (V,A)$ and $s,t \in V$ and degree-$d$ interaction costs $q$.
    %Consider a fixed system $N \subseteq A$ of nonbasic arcs. 
  The instance $I^{(a)}$ of the APECP$_{d-1}$ corresponding to the strongly basic arc $a=(u,v)$ takes as input   the digraph $G^{(a)} = (V_u, E_u)$ with  source vertex $s' = s$,  sink vertex $t' = u$, where 
 $V_u$ is the set of vertices in $V$ lying on at least one $s$-$u$-path and $A_u$  is the set of arcs in $A$ lying on at least one $s$-$u$-path.
  The  order-$(d-1)$  cost function   $q_{d-1}^{(a)}\colon \{ B \subseteq A_u\colon  |B| \leq d-1\} \to \rz$ is given by
    \begin{equation} \label{eq:definition_q_e}
        q_{d-1}^{(a)}(B) := \left(\sum_{\substack{C \subseteq N_u\\ |C| \leq d - |B|}}q_d(B \cup C) \right) - \left( \sum_{\substack{C \subseteq a \cdot N_v\\ |C| \leq d - |B|}}q_d(B \cup C) \right). 
    \end{equation}
\end{definition}

\begin{lemma}
\label{lemma:corresponding-instance}
Let $d \geq 2$ and let  $(G, s,t, q_d)$ be an instance of the \textsc{Lin}SPP$_d$ with a fixed system of nonbasic arcs  $N$. 
The APECP$_{d-1}$ instance $I^{(a)}$ corresponding to some strongly basic arc $a$ is a \textsc{YES}-instance iff the arc $a$ has the property $(\pi)$ with respect to $f\colon \Pst\to\rz$ given by $f(P)=SPP_d(P,q_d)$ for $P\in \Pst$. In this case, $\xval(a) = \beta_a$, where $\beta_a$ is the common cost of all paths in the APECP$_{d-1}$ instance $I^{(a)}$.
\end{lemma}
\begin{proof}
Let $a = (u,v) \in A$ be a strongly basic arc and let $P$ be some $s$-$u$-path in $G$. Then $P$ is  contained in the graph $G^{(a)} = (V_u, A_u)$. 
Let $f^{(a)}(P)=SPP_{d-1}(P,q^{(a)}_{d-1})$ for any $s$-$u$-path $P$ in $G$. We have that
\begin{align*}
    &\quad \xval(a, P) = f_q(P \cdot N_u) - f_q(P \cdot a \cdot N_v) \\
    &= \sum_{\substack{F \subseteq P \cdot N_u\\ |F| \leq d}}q_d(F) - \sum_{\substack{F \subseteq P \cdot a \cdot N_v\\ |F| \leq d}}q_d(F)\\
    &= \sum_{k=0}^d \sum_{\substack{B \subseteq P\\ |B| = k}}\sum_{\substack{C \subseteq N_u\\ |C| \leq d - k}}q_d(B \cup C) - \sum_{k=0}^d \sum_{\substack{B \subseteq P\\ |B| = k}}\sum_{\substack{C \subseteq a \cdot N_v\\ |C| \leq d - k}}q_d(B \cup C)\\
    &= \sum_{k=0}^{d-1} \sum_{\substack{B \subseteq P\\ |B| = k}}\sum_{\substack{C \subseteq N_u\\ |C| \leq d - k}}q_d(B \cup C) - \sum_{k=0}^{d-1} \sum_{\substack{B \subseteq P\\ |B| = k}}\sum_{\substack{C \subseteq a \cdot N_v\\ |C| \leq d - k}}q_d(B \cup C) \quad +  (1 - 1)\sum_{\substack{B \subseteq P\\ |B| = d}}q_d(B)\\
     &= \sum_{k=0}^{d-1} \sum_{\substack{B \subseteq P\\ |B| = k}}\left(\sum_{\substack{C \subseteq N_u\\ |C| \leq d - k}}q_d(B \cup C) - \sum_{\substack{C \subseteq a \cdot N_v\\ |C| \leq d - k}}q_d(B \cup C)
     \right)\\
     &= \sum_{\substack{B \subseteq P\\ |B| \leq d-1}}q^{(a)}_{d-1}(B) = f^{(a)}(P).
\end{align*}
We conclude that the value $\xval(a, P)$ is independent of $P$, if and only if for every path the quantity $f^{(a)}(P)$ does not depend on $P$. The latter condition is equivalent to  $I^{(a)}$  being a \textsc{YES}-instance of the APECP$_{d-1}$. Furthermore, if this is the case, then $\xval(a) = f^{(a)}(P)$ for any $s$-$u$-path $P$.
\qed
\end{proof}
\cref{lemma:property-pi,lemma:corresponding-instance} imply that an instance $(G,s,t,q_d)$ of the SPP$_d$ with an acyclic digraph $G$ is linearizable  iff each  instance $I^{(a)}$ of the APECP$_{d-1}$ corresponding to some strongly basic arc $a$   (with respect to some fixed system of nonbasic arcs)  is a YES-instance. Furthermore, the same lemmas imply that in this case a linearizing cost function in reduced form is obtained by letting $c(a) = \beta_a$ for all strongly basic arcs, $c(a) = 0$ for all nonbasic arcs, and $c(a) = \text{SPP}_d(a \cdot N_v)$ for all arcs $a = (s,v)$ incident to the source.


Thus, we have shown that an instance of the  \textsc{Lin}SPP$_d$ can be reduced to $\bigO(m)$ instances  of the APECP$_{d-1}$.  
Next, in Lemma~\ref{lemma:reduction-APEC} we show that each instance of the APECP$_{d-1}$  can  be reduced to an instance of  the \textsc{Lin}SPP$_{d-1}$.
%
First, we define a specific cost function as follows. 
Let $G = (V, A)$ be a $\Pst$-covered acyclic digraph and $\beta \in \R$. The function $\const_\beta : A \rightarrow \R$  assigns cost $\beta$ to every arc incident to the source $s$, and $0$ to all other arcs.

\begin{lemma}
\label{lemma:reduction-APEC}
    Let $G = (V, A)$ be a $\Pst$-covered acyclic digraph with source vertex $s$ and sink vertex $t$ and let  $N \subseteq A$ a fixed system of nonbasic arcs. Let $q_d$ be an order-$d$  cost function. The instance $(G,s,t,q_d)$ of the APECP$_d$ problem is a YES-instance iff the instance $(G,s,t, q_d)$ of SPP$_d$ is linearizable and $\const_\beta$ is its unique linearizing function in reduced form (with respect to $N$).
\end{lemma}
\begin{proof}
    Clearly,  $\const_\beta$ is a linearizing function iff   all paths have the same cost $\beta$. % Further, if all paths have the same cost $\beta$, then  $\const_\beta$ is a linearizing function. 
    Furthermore, observe that all  arcs incident to the source do not belong to $N$. Therefore $\const_\beta$ is in reduced form with respect to $N$. In fact, by \cref{lemma:nonbasic-arcs}  $\const_\beta$ is the unique linearizing  functions in reduced form, and $\red(c') = \const_\beta$ for all other linearizing functions $c'$.
\qed
\end{proof}


Finally, we show in \cref{lemma:reduction-APEC} how the APECP$_1$ can be computed directly.
\begin{lemma}
\label{lemma:apec-1}
    The APECP$_1$ can be solved in linear time $\bigO(m)$.
\end{lemma}
\begin{proof}
    This is an easy exercise in dynamic programming. The algorithm uses the fact that in a $\Pst$-covered acyclic digraph all $s$-$t$-paths have the same cost if and only for every vertex $w$ all $s$-$w$-paths have the same cost. Hence we can introduce a variable $y_w \in \R$ for every vertex $w$. We let $y_s = 0$ and then check in topological vertex order for every vertex $w$, whether the value $y_u + q_1(\set{(u,w)})$ is the same for every incoming edge $(u, w)$. Finally, if this is the case, the returned common cost of all paths is given by $y_t + q_1(\emptyset)$.
    
    (We remark that attention to the case $q_1(\emptyset) \neq 0$ is crucial. This is because by definition, the term $q_1(\emptyset)$ is always included in SPP$_1(P, q_1)$ for every possible path $P$. Indeed, \cref{eq:definition_q_e} may produce instances with the property $q_1(\emptyset) \neq 0$.)
    \qed
\end{proof}

%%%%%%%%%%%%%%%%%%%%%%%%%%%%%%%%%%%%%%%%%%%%%%%%%%%%%%%%%%%%%%%%%%%%%%%%%%%

\subsection{The linear time \textsc{Lin}SPP$_d$ algorithm}
\label{subsection:alg}
Our \textsc{Lin}SPP$_d$ algorithm ${\cal A}$ works as follows.
Consider an instance $(G,s,t,q_d)$ of the \textsc{Lin}SPP$_d$ with an acyclic $\Pst$-covered digraph $G$, with source vertex $s$, sink vertex $t$ and order-$d$ cost function $q_d$. We first fix some system of nonbasic arcs $N$ and construct the instance $I^{(a)}$ of the  APECP$_{d-1}$ problem given in \cref{def:corresponding-instance} for each strongly basic arc $a$. 
Then,  we  check each instance  $I^{(a)}$ for being a  \textsc{YES}-instance and
do this by reducing  $I^{(a)}$  to  an instance of \textsc{Lin}SPP$_{d-1}$ according to \cref{lemma:reduction-APEC}.
By iterating this process we eventually end up with APECP problems of order $1$ that can be easily solved by  dynamic programming, as demonstrated in \cref{lemma:apec-1}. A summary in pseudocode is provided in Algorithm~\ref{alg}. The correctness of the algorithm follows from \cref{lemma:property-pi,lemma:corresponding-instance,lemma:reduction-APEC,lemma:apec-1}.

\begin{algorithm}[htpb]
    \SetKwFunction{FLinearizable}{Linearizable}
    \SetKwFunction{FAPEC}{APECP}
    \SetKwProg{Fn}{Function}{:}{}
 \Fn{\FLinearizable{$G$,$q$,$d$}}{
 \KwIn{acylic $\Pst$-covered $G$; integer $d \geq 2$; order-$d$ interaction costs $q_d$.}
 \KwOut{$\False$ if $(G, q_d)$ is not linearizable, otherwise a tuple $(\True, c)$ such that $c$ is a linearizing cost function in reduced form.}
 $N \gets$ some system of nonbasic arcs\;
 Calculate $\set{(G^{(a)}, q^{(a)}_{d-1}) : a \text{ strongly basic}}$ \tcp*{using \cref{lemma:faster-reduction}.} \label{alg:linenumber:3}
 \eIf{\FAPEC{$G^{(a)}$, $q^{(a)}_{d-1}$, $d-1$} = {\False} for some strongly basic $a$}{
    \KwRet \False\;
 }{
  We have \FAPEC{$G^{(a)}$, $q^{(a)}_{d-1}$, $d-1$} = $(\True, \beta_a)$ for all strongly basic $a$\\
    $c(a) \gets
    \begin{cases}
        \beta_a; & \text{$a$ strongly basic. }\\
        f_q(a \cdot N_v) & \text{$a = (u,v)$ incident to source} \label{alg:linenumber:f_q} \tcp*{using \cref{lemma:faster-reduction}.}\\
        0; &  \text{$a$ nonbasic.}
    \end{cases}$\;
    \KwRet $(\True, c)$\;
 }
 }
 
 \Fn{\FAPEC{$G$,$q$,$d$}}{
  \KwIn{acylic $\Pst$-covered $G$; integer $d \geq 1$; order-$d$ interaction costs $q_d$.}
 \KwOut{$\False$ if not all source-sink paths in $(G, q_d)$ have same cost, otherwise a tuple $(\True, \beta)$ such that all paths have cost $\beta$.}
 \uIf{$d=1$}{
    solve directly\;
 }
 \uElseIf{\FLinearizable{$G$,$q_d$,$d$} = \False}{
    \KwRet \False\;
 }
 \uElse{
    We have \FLinearizable{$G$,$q_d$,$d$} = $(\True, c)$ for some $c$\;
    \eIf{$c = \const_\beta$ for some $\beta$}{
        \KwRet $(\True, \beta)$\;
    }{
        \KwRet \False\;
    }
 }
 }
 \caption{An algorithm to solve the \textsc{Lin}SPP$_d$.}
 \label{alg}
\end{algorithm}


\label{subsection:alg-runtime}
It is not hard to implement the algorithm described above
in $\bigO(d^2m^{d+1})$ time. However, we can do better. With a careful implementation it is possible to achieve a better result.

\begin{theorem}
\label{thm:alg-linear-time}
The \textsc{Lin}SPP$_d$ on acyclic digraphs can be solved in $\bigO(d^2m^d)$ time.
\end{theorem}

Note that  the input size required to encode the cost function $q_d$ equals $\sum_{k=0}^d \binom{m}{k} \geq m^d / d!$. Thus, $\bigO(d^2m^d)$ is linear in the input size and hence  optimal if  $d$ is considered a constant, like for example in the QSPP.
The remainder of the section is devoted to prove \cref{thm:alg-linear-time}. The main bottleneck we have to get rid of is the computation of the instances $I^{(a)}$ corresponding to the strongly basic arcs $a$ in line~\ref{alg:linenumber:3} of Algorithm~\ref{alg}. If one simply uses the definition of $q^{(a)}_{d-1}$ from \cref{eq:definition_q_e}, then one can see that for each $a$ one can compute $q^{(a)}_{d-1}$ in $\bigO(dm^d)$ time. As this needs to be repeated for each strongly basic arc $a$, in total this would take $\bigO(dm^{d+1})$ time. A similar bottleneck arises in line~\ref{alg:linenumber:f_q} when computing $f_q(a \cdot N_v)$. We get rid of these two bottlenecks by calculating some \emph{helper values}.

For all sets $B \subseteq A$ of arcs with $|B| \leq d-1$ and all vertices $x \in V \setminus \set{s}$, we define the helper value
\begin{equation}
    \gamma(B, x) := \sum_{\substack{C \subseteq N_x\\ |C| \leq d - |B|}}q_d(B \cup C). \label{eq:helper-values}
\end{equation}


 
\begin{lemma}
\label{lemma:faster-reduction}
The line~\ref{alg:linenumber:3} and the line~\ref{alg:linenumber:f_q} of Algorithm~\ref{alg} can be implemented such that their execution takes at most $c d m^d$ steps for some constant $c \geq 0$. This constant is independent of both $n$ and $d$.
\end{lemma}
\begin{proof}
    Consider the helper values $\gamma(B,x)$ from \cref{eq:helper-values}.
    We show how to precompute all the values of $\gamma$ for $B \subseteq A, |B| \leq d-1, x \in V \setminus \set{s}$ in $\bigO(dm^d)$ time. We begin by showing for some fixed set $B \subseteq A$ with $|B| \leq d-1$, how to compute the value $\gamma(B, x)$ for all vertices $x \in V\setminus\set{s}$. Let $k = |B|$ be the size of $B$, we have $k \in \fromto{0}{d-1}$. Assume that there is some nonbasic arc $a = (u, v) \in N$ such that we already computed $\gamma(B, v)$ and now want to compute $\gamma(B,u)$. Note that $N_u = a \cdot N_v$, because $a \in N$. We have the recursive formula
    \begin{align}
        \gamma(B, u) &= \sum_{\substack{C \subseteq a \cdot N_v\\ |C| \leq d - |B|}}q_d(B \cup C) \nonumber\\
        &=  \sum_{\substack{C \subseteq N_v\\ |C| \leq d - |B| - 1}}q_d(B \cup \set{a} \cup C) + \sum_{\substack{C \subseteq N_v\\ |C| \leq d - |B|}}q_d(B \cup C)\nonumber\\
        &=  \gamma(B, v) + \sum_{\substack{C \subseteq N_v\\ |C| \leq d - |B| - 1}}q_d(B \cup \set{a} \cup C). \label{eq:recursive-gammma}
    \end{align}
    This formula can be evaluated in $\binom{m}{{d - k - 1}}$ time. We can now traverse the tree of nonbasic arcs, starting at the root $t$, where $\gamma(B, t) = q(B)$, and iteratively apply the formula until we obtained all the values $\gamma(B, x)$ for the vertices $x \in V \setminus \set{s}$.
    In total, it takes $n\binom{m}{{d - k - 1}}$ time to compute all values $\gamma(B, x)$ for a fixed set $B$ of size $k$. For each $k = 0,\dots,d-1$, there are $\binom{m}{k}$ subsets of size $k$. Therefore, the total time to compute all values of $\gamma$ is
    \begin{align*}
        \sum_{k=0}^{d-1} n\binom{m}{d - k - 1}\binom{m}{k} \leq \sum_{k=0}^d n\frac{m^{d-k-1}}{(d-k-1)!} \frac{m^k}{k!} \leq (d+1) nm^{d-1}.
    \end{align*}
    Here we used the very rough estimate $t! \geq 1$. We conclude that at most $c_1dm^d$ steps are necessary to precompute the values of $\gamma$ for some constant $c_1 \geq 0$.
    
    Now assume that all the values $\gamma(B, x)$ have been precomputed. For every strongly basic arc $a = (u, v)$, and each set $B \subseteq A_u$ with $|B| \leq d-1$, the interaction cost of $B$ in the corresponding APECP$_{d-1}$ instance is given by
    \begin{align*}
        q_{d-1}^{(a)}(B) = \gamma(B,u) - \gamma(B, v) - \sum_{\substack{C \subseteq N_v\\ |C| \leq d - |B| - 1}}q_d(B \cup \set{a} \cup C).
    \end{align*}
    This can be seen by plugging the definition of $\gamma$ into \cref{eq:definition_q_e}. This equation can be evaluated in $\binom{m}{d - k -1}$ time if $k=0,\dots,d-1$ is the size of $B$. In order to obtain all the desired values for $q^{(a)}_{d-1}(B)$ for all the APECP$_{d-1}$ instances, we need to consider every choice of the set $B$ (again there are at most $\binom{m}{k}$ sets of size $k$) and every of the at most $\bigO(m)$ choices of the arc $a$. 
    The same analysis as before shows that given the precomputed $\gamma$-values, all required values of $q^{(a)}_{d-1}(B)$ can be computed in at most $c_3dm^d$ steps for some constant $c_3 \geq 0$. Finally, computing the vertex set $V_u$ and arc set $A_u$ of the APECP$_{d-1}$ instance clearly can be done in linear time $\bigO(m)$ for each strongly basic arc $a = (u,v)$. We conclude that computing the set $\set{(G^{(a)}, q^{(a)}_{d-1}) : a \text{ strongly basic })}$ of all APECP$_{d-1}$ instances can be done in at most $c_4dm^d$ steps for some constant $c_4 \geq 0$.

    
    Finally consider line~\ref{alg:linenumber:f_q} of Algorithm~\ref{alg} and assume that all helper values $\gamma(B,x)$ have been precomputed. Let $a = (s,v)$ be an arc incident to the source. We have that
    \begin{align*}
        f_q(a \cdot N_v) = \gamma(\emptyset, v) + \sum_{\substack{C \subseteq N_v\\ |C| \leq d - 1}}q(\set{a} \cup C).
    \end{align*}
    This formula can be evaluated in $\bigO(m^{d-1})$ time for each arc $a$. We conclude that the set of all values $f_q(a \cdot N_v)$ in line~\ref{alg:linenumber:f_q} of Algorithm~\ref{alg} can be computed in at most $c_5m^d$ steps for some constant $c_5 \geq 0$.
    \qed
\end{proof}

\begin{proof}[Proof of \cref{thm:alg-linear-time}]
It follows from \cref{lemma:property-pi,lemma:corresponding-instance,lemma:reduction-APEC,lemma:apec-1} that Algorithm~\ref{alg} correctly solves the \textsc{Lin}SPP$_d$. It follows from \cref{lemma:faster-reduction} that lines \ref{alg:linenumber:3} and \ref{alg:linenumber:f_q} can be implemented to take at most $c_0dm^d$ steps each for some constant $c_0 \geq 0$. The running time of all the remaining lines is asymptotically dominated by $m^d$. Now let $f(m, d)$ denote the worst-case running time of $\FLinearizable{}$ when input a digraph on $m$ arcs and degree-$d$ interaction cost. Let $g(m, d)$ be the corresponding function for $\FAPEC{}$. By the preceding arguments, there exists a constant $c \geq 0$ such that the following inequalities hold:
\begin{align*}
    f(m ,d) &\leq cdm^d + mg(m, d-1); & d \geq 2\\
    g(m, d) &\leq cm + f(m, d); & d \geq 2\\
    g(m, 1) &\leq cm.
\end{align*}
We show by induction over $d$ that $f(m, d) \leq cd(2d - 2)m^d$ and $g(m, d) \leq cd(2d - 1)m^d$. Indeed, in the base case $d = 1$ we have $g(m, 1) \leq cm$. For $d \geq 2$ we have 
\begin{align*}
    f(m, d) &\leq cdm^d + mc(d-1)(2d - 3)m^{d-1} \leq cd(2d - 2)m^d \\
    g(m, d) &\leq cm + cd(2d - 2)m^d \leq cd(2d -1)m^d. 
\end{align*}
This implies that $f = \bigO(d^2m^d)$, which was to show.
\qed
\end{proof}

\section{The subspace of linearizable instances}
\label{sec:subspace}

Let $d\in \nz$, $d\ge 2$, and a $\Pst$-covered acyclic digraph  $G = (V, A)$ with source vertex $s$ and sink vertex $t$  be  fixed. 
Let $H^{(d)} := \set{B \subseteq H \mid |B| \leq d}$ be the set of all subsets of at most $d$ arcs in arc set $H\subseteq A$.
Every order-$d$ cost function $q_d\colon A^{(d)}\to \rz$ can be uniquely represented by a vector $x \in \R^{A^{(d)}}$ with $q_d(F)=x_F$ for all $F\in A^{(d)}$, and vice-versa. Thus, each instance $(G,s,t,q_d)$ can be identified with the corresponding vector $x\in \rz^{A^{(d)}}$ and we will say that $x\in \R^{A^{(d)}}$ is an instance of the SPP$_d$. 
It is straightforward to see that if $x,y \in \R^{A^{(d)}}$ are linearizable instances of the SPP$_d$, then $\mu x + \nu y$ is also a linearizable instance, for all scalars $\mu,\nu \in \R$.
Therefore, the set of linearizable instances of the SPP$_d$ on the fixed digraph  $G$ form a linear subspace ${\cal L}_d$ of $\R^{A^{(d)}}$. 

Methods to compute this subspace 
are useful in
B\&B algorithms 
for the SPP$_d$ as they can be applied to compute
better lower bounds along the lines of what Hu and Sotirov~\cite{huSo2021} did for general quadratic binary programs.
Hu and Sotirov showed that for $d = 2$ a basis of ${\cal L}_d$ can be computed in polynomial time \cite[Prop. 5]{huSo2021}.
We extend their result 
to the case  $d > 2$.
\begin{theorem}
\label{thm:subspaces}
Let $G = (V, A)$ be  a $\Pst$-covered, acyclic digraph with source vertex $s$ and sink vertex $t$ and let   $d \in \N$ be a constant.  A basis of the subspace ${\cal L}_d$  of the  linearizable instances of the SPP$_d$  can be computed in polynomial time.  
\end{theorem}


The goal of this setion is to prove \cref{thm:subspaces}. We first explain the main idea of the proof. The idea is to specify  a  $k\in \nz$ and  a  matrix $M$ of 
polynomially bounded dimensions, 
such that for  $f: \R^{A^{(d)}} \rightarrow \R^k$ with  $f(x)=Mx$,  we have:
$f(x) = 0$ 
iff
$x$ is  a linearizable instance of the SPP$_d$. Thus,  the linearizable instances $x$ of the SPP$_d$ 
form the kernel of $M$, which can be efficiently computed.

This function $f$ will be composed of smaller building blocks, mimicking the way that our algorithm from \cref{subsection:alg} reduces the SPP$_d$ to smaller instances of the APECP$_{d-1}$, which in turn is reduced to the SPP$_{d-1}$. Formally, let for a vertex $u$ be the vertex set $V_u$ and the arc set $A_u$ be the set of vertices (arcs) which lie on at least one $s$-$u$-path, as specified in \cref{def:corresponding-instance}. An instance of the SPP$_{d}$ on the smaller digraph $(V_u, A_u)$ can be interpreted as a vector $x \in \R^{A_u^{(d)}}$. Likewise, an instance of the APECP$_{d}$ on the smaller digraph $(V_u, A_u)$ can be interpreted as a vector $x \in \R^{A_u^{(d)}}$. In the following, we show that for all vertices $u \in V$ and every $d \in \N, d\geq 2$ there exists some polynomially bounded $k \in \N$ and a linear function
\[
f_{u,d} : \R^{A_u^{(d)}} \rightarrow \R^k
\]
such that $f_{u,d}(x) = 0$ if and only if $x$ is a YES-instance of the \textsc{lin}SPP$_d$ on the digraph $(V_u, A_u)$. Likewise, we show that for all vertices $u \in V$ and every $d \in \N, d\geq 1$ there exists some polynomially bounded $k \in \N$ and a linear function
\[
g_{u,d} : \R^{A_u^{(d)}} \rightarrow \R^k
\]
such that $g_{u,d}(x) = 0$ if and only if $x$ is a YES-instance of the APECP$_{d}$ on the digraph $(V_u, A_u)$. 

We show these two claims by induction. The base case of the induction is concerned with the APECP$_1$. For the remainder of this section, we consider a fixed $\Pst$-covered digraph $G = (V, A)$ with source vertex $s$ and sink vertex $t$ on $n$ vertices and $m$ arcs.

\begin{lemma}
\label{lemma:subspaces-apec-1}
    Let $u \in V$ be a vertex and consider the APECP$_1$ on the digraph $(V_u, A_u)$. Then the following holds:
    \begin{enumerate}[(i)]
        \item There exists $k \in \N$ and a linear function $g_{u,1} : \R^{A_u^{(1)}} \rightarrow \R^k$, such that $k = \bigO(m)$ and such that $g_{u,1}(x) = 0$ iff $x$ is a YES-instance of the APECP$_1$.
        \item There exists a linear function $g'_{u,1} : \R^{A_u^{(1)}} \rightarrow \R$, such that for all $x$ with $g_{u,1}(x) = 0$, we have that $g'_{u,1}(x)$ is the common cost of the paths in the APECP$_1$ instance $x$.
        \item There is an algorithm computing the functions $g_{u,1}, g'_{u,1}$ in $\bigO(m)$ time.
    \end{enumerate}
\end{lemma}
\begin{proof}
    Assume we are given an instance $x \in \R^{A_u^{(1)}}$. The vector $x$ represents a function $x : A_u^{(1)} \rightarrow \R$. We write $x(F)$ instead of $x_F$ for ease of notation.
    We consider the same dynamic program as in \cref{lemma:apec-1}. For every vertex $w \in V_u$, we fix some arbitrary $s$-$w$-path $P_w$.
    We introduce a helper variable $y_w$ for every vertex $w \in V_u$ by making use of $P_w$.
    \[ y_w := \sum_{a \in P_w}x(\set{a}).\]
    Note that $y_w$ depends only linearly on $x$. Note furthermore that by the same argumentation as in \cref{lemma:apec-1}, the dynamic program correctly concludes that the APECP$_1$ instance is a YES-instance iff
    \[\forall a = (w,z) \in A_u : y_z = y_w + x(\set{a}). \]
    Furthermore, if $x$ is indeed a YES-instance, then the common ocst of all paths equals exactly $y_u + x(\emptyset)$. (Note that indeed the case $x(\emptyset) \neq 0$ can appear.)

    For this reason, we can construct the functions $g_{u,1}, g'_{u,1}$ the following way: The function $g_{u,1}$ maps to $\R^k$ with $k = |A_u|$ and is given by
    \[ g_{u,1}(x) = ( y_w + x(\set{a}) - y_z)_{a = (w,z) \in A_u}.\]
    The function $g'_{u,1}$ is given by $g'_{u,1}(x) = y_u + x(\emptyset)$.

    Note that indeed the two provided functions are linear. By construction we have properties (i) and (ii). Finally, it is straightforward to see that given an input $x \in \R^{A_u^{(1)}}$, the vector $g_{u,1}(x)$ can be computed in $\bigO(m^2)$ time and the value $g'_{u,1}(x)$ can be computed in $\bigO(m)$ time. However, note that we have freedom to choose the paths $P_w$. If we choose the paths $P_w$ cleverly, so that they form an out-tree rooted at $s$, then one can also compute the vector $g_{u,1}(x)$ in $\bigO(m)$ time.  \qed
\end{proof}


\begin{lemma}
\label{lemma:subspaces-induction}
    Let $d \geq 2$, let $u \in V$ be a vertex and consider the  SPP$_d$ and the APECP$_{d}$ on the digraph $(V_u, A_u)$. Then the following holds:
    \begin{enumerate}[(i)]
        \item There exists $k \in \N$ and a linear function $f_{u,d} : \R^{A_u^{(d)}} \rightarrow \R^k$, such that $k = \bigO(m^{d})$ and such that $f_{u,d}(x) = 0$ iff $x$ is a linearizable instance of the SPP$_{d}$.
        \item There exists a linear function $f'_{u,d} : \R^{A_u^{(d)}} \rightarrow \R^{A_u}$, such that for all $x$ with $f_{u,d}(x) = 0$, we have that $f'_{u,d}(x) \in \R^{A_u}$ is a linearizing cost function in reduced form of the SPP$_d$ instance $x$.
        \item There exists $k \in \N$ and a linear function $g_{u,d} : \R^{A_u^{(d)}} \rightarrow \R^k$, such that $k = \bigO(m^{d})$ and such that $g_{u,d}(x) = 0$ iff $x$ is a YES-instance of the APECP$_{d}$.
        \item There exists a linear function $g'_{u,d} : \R^{A_u^{(d)}} \rightarrow \R$, such that for all $x$ with $g_{u,d}(x) = 0$, we have that $g'_{u,d}(x)$ is the common cost of the paths in the APECP$_{d}$ instance $x$.
        \item There is an algorithm computing the functions $f_{u,d}, f'_{u,d}, g_{u,d}, g'_{u,d}$ in $\bigO(m^d)$ time.
    \end{enumerate}
\end{lemma}
\begin{proof}
    We prove the lemma using induction. The base case of the induction is provided by \cref{lemma:subspaces-apec-1}, which proves items (iii) and (iv) in the case $d = 1$. For the inductive step, assume that for some value of $d - 1$ with  $d \geq 2$ we have proven (iii) and (iv), that is we have assured the existence of $g_{u,d-1}$ and $g'_{u,d-1}$. We now proceed to show that items (i) -- (iv) hold for the value $d$, that is, we construct the functions $f_{u,d}, f'_{u,d}, g_{u,d}, g'_{u,d}$.

    In order to construct $f_{u,d}$, consider an instance $q_d \in \R^{A_u^{(d)}}$ of the SPP$_d$ on the graph $(V_u, A_u)$. Fix some system $N \subseteq A_u$ of nonbasic arcs with respect to that graph $(V_u, A_u)$. Let $F \subseteq A_u$ be the set of strongly basic arcs with respect to $N$, and let $F'$ be the set of arcs which are incident to the source (hence $A_u = N \cup F \cup F'$ is a partition by the definition of basic/strongly basic arcs). Now, by \cref{lemma:property-pi,lemma:corresponding-instance} we have that the instance $q_d$ is a YES-instance of \textsc{Lin}SPP$_d$ iff for every arc $a = (w,z) \in F$, we have that $q^{(a)}_{d-1}$ is a YES-instane of APECP$_{d-1}$. By the inductive assumption, the latter is the case iff  $g_{w,d-1}(q^{(a)}_{d-1}) = 0$. Observe that the function which maps $q_d \in \R^{A_u^{(d)}}$ to $q^{(a)}_{d-1} \in \R^{A_w^{(d-1)}}$ is a linear function (this follows from the definition, i.e.\ \cref{eq:definition_q_e}). Hence the composition $g_{w,d-1} \circ\ q^{(a)}_{d-1}$ is again a linear function. Therefore, if we define
    \begin{align*}
        f_{u,d}(q_d) = (g_{w,d-1}(q^{(a)}_{d-1}))_{a = (w,z) \in F}, \\
    \end{align*}
    then $f_{u,d}$ is a linear function and maps to $\R^k$ with $k = \bigO(m \cdot m^{d-1}) = \bigO(m^d)$. By the arguments above, property (i) is satisfied. Furthermore, if we define
    \begin{align*}
        (f'_{u,d}(q_d))(a) = \begin{cases}
            g'_{w,d-1}(q^{(a)}_{d-1}); & a = (w,z) \in F\\
            0; & a \in N\\
            \text{SPP}_d(a \cdot N_z, q_d); & a = (w,z) \in F',
        \end{cases} \\
    \end{align*}
    then $f'_{u,d}(q_d) : A_u \rightarrow \R$ is a linearizing cost function whenever $f_{u,d}(q_d) = 0$. Furthermore this function is in reduced form. This follows from the arguments in \cref{lemma:property-pi,lemma:corresponding-instance} and the inductive assumption on $g'_{w,d-1}$. 
    The function $f'_{u,d}$ is linear. 
    It follows from the speedup technique from \cref{subsection:alg} that given $q_d$, one can compute the sequence of values $(q^{(a)}_{d-1})_{a \in F}$, as well as the sequence $(\text{SPP}_d(a \cdot N_z; q_d))_{a\in F}$ in $\bigO(m^d)$ time. 
    Combined with the inductive assumption that both $g_{w,d-1}, g'_{w,d-1}$ can be computed in $\bigO(m^{d-1})$ time for each vertex $w$, we obtain that both $f_{u,d}, f'_{u,d}$ can be computed in $\bigO(m^d)$ time. Hence we have shown (i) and (ii).

    Secondly, assuming that (i) and (ii) hold, we show how to construct $g_{u,d}, g'_{u,d}$. Let $q_d \in \R^{A_u^{(d)}}$ be an instance of the APECP$_d$. It follows by the arguments from \cref{lemma:reduction-APEC} that $q_d$ is a YES-instance where all paths have cost $\beta$ iff $q_d$ is linearizable and for the linearizing function $c$ it holds that $\red(c) = \const_\beta$. By the inductive assumption and the uniqueness of the reduced form, this is the case iff 
    \begin{align*}
        &f_{u,d}(q_d) = 0 \\
        \text{ and } &\forall a,a' \in F':  (f'_{u,d}(q_d))(a) = (f'_{u,d}(q_d))(a') \\
        \text{ and } &\forall a \in A_u \setminus F': (f'_{u,d}(q_d))(a) = 0. 
    \end{align*}
    Similarly to the argument above, we can construct a linear function $g_{u,d}$ which maps to $\R^k$ for some $k = \bigO(m^d)$ such that $g_{u,d} = 0$ iff all three of the above conditions are satisfied. By choosing an arbitrary arc $a \in F'$ we can define $g'_{u,d}(q_d) := (f'_{u,d}(q_d))(a)$. Finally, by the inductive assumption, we can compute $f_{u,d}(q_d), f'_{u,d}(q_d)$ in $\bigO(m^d)$ time. It follows that the functions $g_{u,d}(q_d), g'_{u,d}(q_d)$ can be computed in $\bigO(m^d)$ time. Hence we have shown (iii), (iv) and (v). \qed
\end{proof}

\begin{proof}[Proof of \cref{thm:subspaces}]
    Consider the sink vertex $t$. By \cref{lemma:subspaces-induction}, the function $f_{t,d}$ has the property that some instance $q_d \in \R^{A_u^{(d)}}$ is linearizable iff $f_{t,d}(q_d) = 0$. 
    We show how to compute a matrix representation $M$ of this function, such that $f_{t,d}(x) = Mx$ for all $x$. Hence $\mathcal{L}_d$ is equal to the kernel of $M$.
    Consider the set $e_1,\dots, e_k$ of all standard basis vectors of $\R^{A_u^{(d)}}$, where $k = \bigO(m^d)$. By \cref{lemma:subspaces-induction}, we can compute the vector $f_{t,d}(e_j)$ in $\bigO(m^d)$ time for every $j=1,\dots,k$. Basic linear algebra tells us that these vectors constitute the columns of $M$. We can hence compute the matrix $M$ and a basis of its kernel in polynomial time. We have therefore obtained a basis of $\mathcal{L}_d$. \qed
\end{proof}

\noindent \textbf{ Acknowledgement.} This research has been supported by the Austrian Science Fund (FWF): W1230.
\section{Conclusions and open questions}\label{conclu:sec}
 


\bibliographystyle{plain}
\bibliography{literature.bib}



\end{document}


