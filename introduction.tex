\pagenumbering{arabic}
\chapter{Introduction}
Less than 80 years after the invention of the first computer, today we are surrounded by digital technology at every step we make. 
Computers influence and control a huge amount of aspects of modern life. 
We have grown so accustomed to computers, that we take many of their wonderous abilities for granted. 
One of these magical abilities is to find the optimal solution to a problem out of an incredibly large amount of possibilities.
For example, suppose you wanted to go from Paris to Berlin by car. There is an almost infinite amount of different paths from Paris to Berlin. Yet, a clever computer algorithm can select the single unique path which is the fastest among all of them.

This ability of computers to find the optimal solution for a given problem is used in many areas of today's live: Computers are used to find the cheapest flight schedule for an airline, to find the best investment scheme for a portfolio, to decide which taxis from a taxi company should pick up which customer, to design optimal communication networks, and many, many more problems. Application areas range from Economics, Logistics, Operations Research, Computer Science, Healthcare, Biology, and many other disciplines.

It is important to state that computers do not come with this ability a priori. Instead, specific programs and algorithms need to be developed, to be able to handle the huge amount of possibilities. 
\emph{Combinatorial Optimization} is the scientific field concerned with the question: How do we pick the optimal solution out of a huge (but still finite) amount of possibilities? 
In particular, Combinatorial Optimization tries to understand, what all the previously listed problems have in common, and tries to develop a mathematical theory of these problems and the tools to solve them. Classically, Combinatorial Optimization tries to classify problems as either being intractable (NP-hard) or tractable (polynomial-time solvable). For the intractable problems, it tries to understand what exactly makes them intractable, and whether we can find at least approximate, almost-optimal solutions.

The area of Combinatorial Optimization lies in the intersection between discrete mathematics and theoretical computer science. 
Just like the computer, it is a relatively new scientific field (at least compared to other ares of mathematics). In this thesis, we are concerned 

\section{Motivation and background}
Very motivating.
