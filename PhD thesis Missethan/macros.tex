%---Macros-Introduction---------------------------------------------------------------


\newcommand{\TBD}{\textbf{To be done}}

\newcommand{\rounddown}[1]{\left\lfloor#1\right\rfloor} %floor function
\def\ur{\in_R} %element chosen uniformly at random

%Graph properties
\newcommand{\numberVertices}[1]{v\left({#1}\right)} %number of vertices in a graph
\newcommand{\vertexSet}[1]{V\left(#1\right)} %vertex set of a graph
\newcommand{\edgeSet}[1]{E\left(#1\right)} %edge set of a graph
\newcommand{\numberEdges}[1]{e\left({#1}\right)} %number of edges in a graph

\newcommand{\maxdegree}[1]{\Delta\left(#1\right)} %maximum degree
\def\class{\mathcal{A}}
\def\crg{A} %constrained random graph

\DeclareMathOperator{\Cv}{cv} %fraction of cut vertices in a graph
\newcommand{\cv}[1]{\Cv\left(#1\right)} %fraction of cut vertices in a graph

\DeclareMathOperator{\Diam}{diam} %diameter of a graph
\newcommand{\diam}[1]{\Diam\left(#1\right)} %diameter of a graph

%Probability
\newcommand{\prob}[1]{\mathbb{P}\left[#1\right]} %probability
\newcommand{\expec}[1]{\mathbb{E}\left[#1\right]} %expectation
\def\N{\mathbb{N}} %natural numbers

%Asymptotic notation
\newcommand{\smallo}[1]{o\left(#1\right)}
\newcommand{\bigo}[1]{O\left(#1\right)}
\newcommand{\OP}[1]{O_p\left(#1\right)}
\newcommand{\smallomega}[1]{\omega\left(#1\right)}
\newcommand{\OmegaP}[1]{\Omega_p\left(#1\right)}
\newcommand{\ThetaP}[1]{\Theta_p\left(#1\right)}
\newcommand{\Th}[1]{\Theta\left(#1\right)}

%Rooted graphs and limits
\newcommand{\rootedGraph}[2]{\left(#1, #2\right)}
\newcommand{\distributionalLimit}[2]{#1~\xrightarrow{~d~}~#2}
\DeclareMathOperator{\GWT}{GWT}
\newcommand{\gwt}[1]{\GWT\left(#1\right)}
\def\skeletonTree{T_\infty}
\newcommand{\ball}[3]{B_{#1}\left(#2, #3\right)}
\newcommand{\expansion}[3]{\lambda_{#1}\left(#2, #3\right)}
\newcommand{\expansionP}[2]{\lambda_{#1}\left(#2\right)}

%Random graph process
\newcommand{\pro}[2]{P_{#1,#2}} %graph process (graph after #2 edges have been tested)
\newcommand{\acc}[2]{P_{#1,m=#2}} %graph process (graph after #2 edges have been accepted)
\newcommand{\er}[2]{G_{#1,#2}} %graph process (graph after #2 edges have been considered and all of them were added)

%Names
\def\Erdos{Erd\H{o}s}
\def\Renyi{R\'enyi}
\def\Luczak{\L{}uczak}
\def\ER{\Erdos-\Renyi}
\def\Bollobas{Bollob\'{a}s}
\def\GW{Galton-Watson}


%---Macros-Cycles_Blocks---------------------------------------------------------------
\newcommand{\proofof}[1]{\subsection{Proof of \Cref{#1}}}
\newcommand{\proofofL}[1]{\subsection{Proof of \Cref{#1}}\label{CBproof:#1}}
\newcommand{\proofofN}[2]{\subsection[#1]{#1: Proof of \Cref{#2}}}
\newcommand{\proofofLN}[2]{\subsection[#1]{#1: Proof of \Cref{#2}}\label{CBproof:#2}}
\DeclareMathOperator{\cov}{Cov}
%\newcommand{\N}{\mathbb{N}} %natural numbers
\newcommand{\R}{\mathbb{R}} %real numbers
\newcommand{\whp}{whp}%with high probability}
\newcommand{\Whp}{Whp}%Beginning of a sentence

%\newcommand{\prob}[1]{\mathbb{P}\left[#1\right]} %Probability
\newcommand{\condprob}[2]{\mathbb{P}\left[#1 \;\middle|\; #2\right]}
\newcommand{\variance}[1]{\mathbb{V}\left[#1\right]}
\newcommand{\covariance}[2]{\cov\left[#1, #2\right]}
%\newcommand{\expec}[1]{\mathbb{E}\left[#1\right]} %Expectation
%\newcommand{\rounddown}[1]{\left\lfloor#1\right\rfloor} %floor function

\def\Erdos{Erd\H{o}s}
\def\Renyi{R\'enyi}
\def\ER{\Erdos-\Renyi}
\def\nc{N}
\def\nd{k}
\def\val{x}
\def\valSecond{y}
\def\ind{i}
\def\indSecond{j}
\def\fixGraph{H}
\def\fixMultigraph{H}
\def\randomGraph{A}
\def\ur{\in_R}
\def\minimum{{X_*}}
\def\maximum{{X^*}}
\def\LargestComponent{L_1}
\def\planarClass{\mathcal{P}}
\def\planarRandomGraph{P}
\def\randomCore{C}
\def\randomMultiCore{\tilde{C}}
\def\constant{M}
\def\generalGraphClass{\mathcal{A}}
\def\generalRandomGraph{A}
\def\colour{F}
\def\pl{kernel-stable}
\def\Pl{Kernel-stable}
\def\loopInsertion{loop insertion}
\def\loopInsertions{loop insertions}
\def\LoopInsertion{Loop insertion}
\def\cl{\mathcal{A}}
\def\property{\mathcal{Q}}
\def\2s{2-simple}
\def\ce{\alpha}
\def\rest{R}
\def\bridgeInsertion{bridge insertion}
\def\bridgeInsertions{bridge insertions}
\def\BridgeInsertion{Bridge insertion}
\def\bridgeClass{\mathcal{M}}
\def\bridgeMarkedClass{\mathcal{M}'}
\def\bridgeGraph{M}
\def\bridgeStable{bridge-stable}
\def\BridgeStable{Bridge-stable}
\def\bridgeNumber{bridge number}
\def\BridgeNumber{Bridge number}
\def\bridgeNumbers{bridge numbers}
\def\block{B}
\def\bridge{e}
\def\largeBlock{dominant}
\def\bb{b}
\def\wb{w}
\def\condRandomGraph{F}
\def\exponent{\mu}
\def\func{\Phi}
\def\seq{\mathbf{s}}

\newcommand{\girth}[1]{g\left({#1}\right)}
\newcommand{\longestCycle}[1]{c\left({#1}\right)}
\newcommand{\numberLoops}[1]{\lambda\left(#1\right)}
\def\NumberLoops{\lambda}
\newcommand{\numberBridges}[1]{b\left(#1\right)}
\def\NumberBridges{b}
\newcommand{\Tp}[1]{\Theta_p\left(#1\right)}
\newcommand{\Op}[1]{O_p\left(#1\right)}
\newcommand{\Omp}[1]{\Omega_p\left(#1\right)}
\newcommand{\T}[1]{\Theta\left(#1\right)}
\newcommand{\Om}[1]{\Omega\left(#1\right)}
\newcommand{\weight}[1]{w\left(#1\right)}
%\newcommand{\numberVertices}[1]{v\left({#1}\right)}
%\newcommand{\numberEdges}[1]{e\left({#1}\right)}
\newcommand{\core}[1]{C\left(#1\right)}
\newcommand{\kernel}[1]{K\left(#1\right)}
\newcommand{\complexPart}[1]{Q\left(#1\right)}
\newcommand{\largestComponent}[1]{L_1\left(#1\right)}
%\newcommand{\vertexSet}[1]{V\left(#1\right)}
%\newcommand{\edgeSet}[1]{E\left(#1\right)}
\newcommand{\Rest}[1]{R\left(#1\right)}
\newcommand{\bn}[1]{\beta\left(#1\right)}
\newcommand{\Bn}[2]{\beta_{#1}\left(#2\right)}
\newcommand{\Nb}[2]{\beta\left(#1, #2\right)}
\newcommand{\blockOrder}[2]{b_{#1}\left(#2\right)}
\newcommand{\blockLargest}[2]{B_{#1}\left(#2\right)}
\newcommand{\subdivisionNumber}[1]{S\left(#1\right)}
\newcommand{\condGraph}[2]{#1 \mid #2}
\definecolor{lightGray}{RGB}{220,220,220}


%---Macros-Cycles_Blocks------------------------------------------------------------

%\DeclareMathOperator{\Cv}{cv}
%\newcommand{\cv}[1]{\Cv\left(#1\right)}
%\def\ur{\in_R} %element chosen uniformly at random
%\newcommand{\smallo}[1]{o\left(#1\right)}
%\newcommand{\bigo}[1]{O\left(#1\right)}
%\newcommand{\smallomega}[1]{\omega\left(#1\right)}
%\def\N{\mathbb{N}}
%\newcommand{\prob}[1]{\mathbb{P}\left[#1\right]}    %Probability
%\def\Erdos{Erd\H{o}s}
%\def\Renyi{R\'enyi}
%\def\ER{\Erdos-\Renyi}
%\def\Luczak{\L{}uczak}
\DeclareMathOperator{\po}{Po}
\newcommand{\poisson}[1]{\po\left(#1\right)}
%\newcommand{\condprob}[2]{\mathbb{P}\left[#1 \;\middle|\; #2\right]}



%---Macros-Max_Degree-----------------------------------------------------------

%\newcommand{\whp}{whp}%with high probability}
%\newcommand{\Whp}{Whp}%Beginning of a sentence
%\newcommand{\prob}[1]{\mathbb{P}\left[#1\right]} %Probability
\newcommand{\probLarge}[1]{\mathbb{P}\big[#1\big]} %Probability with large parentheses
%\newcommand{\condprob}[2]{\mathbb{P}\left[#1 \;\middle|\; #2\right]}
%\newcommand{\variance}[1]{\mathbb{V}\left[#1\right]}
%\newcommand{\expec}[1]{\mathbb{E}\left[#1\right]} %Expectation

%\def\Erdos{Erd\H{o}s}
%\def\Renyi{R\'enyi}
%\def\Luczak{\L{}uczak}
%\def\ER{\Erdos-\Renyi}
%\def\Bollobas{Bollob\'{a}s}


%\newcommand{\proofof}[1]{\subsection{Proof of \Cref{#1}}}

%\def\N{\mathbb{N}} %natural numbers
\def\R{\mathbb{R}} %real numbers
%\def\ur{\in_R} %element chosen uniformly at random
%\newcommand{\rounddown}[1]{\left\lfloor#1\right\rfloor} %floor function

\def\twoconcentration{D}

%%%%%%%%%%%%%%%%%%%%%%%%%%%%%%%%%%%%%%%%%%%%%%%%%%%%%%%
%Balls into bins

\def\bin{\mathcal{B}}
%\def\ball{B}
\def\nbins{n} %number of bins
\def\nballs{k} %number of balls
\def\location{\mathbf{A}} %location vector
\def\locationBit{A} %coordinate of a location vector
\def\loadvector{\mathbf{\lambda}} %vector of loads
\def\load{\lambda} %load of a bin, i.e. number of balls in the bin
\def\maxload{\lambda^\ast} %maximum load
\DeclareMathOperator{\BB}{BB}
\DeclareMathOperator{\MBB}{M}
\newcommand{\binsandballs}[2]{\BB\left(#1, #2\right)}
\newcommand{\maxbinsandballs}[2]{\MBB\left(#1, #2\right)}

\def\maxloadsubset{\nu^\ast} %maximum load after a certain number of balls

\newcommand{\concentration}[2]{\nu\left(#1, #2\right)} %unique solution of equation k^x*e^x/(x^(x+1/2)n^(x-1))=1
\newcommand{\specialconcentration}[1]{\nu\left(#1\right)} %unique solution of equation above for the special case k=n
\def\Concentration{\nu}

%%%%%%%%%%%%%%%%%%%%%%%%%%%%%%%%%%%%%%%%%%%%%%%%%%%%%%%
%Forests of trees with specified roots
\def\ntrees{t} %number of trees
\def\rootrv{Z} 
\def\rootvector{\mathbf{Y}} 
\def\rootvectorbit{Y} 


%%%%%%%%%%%%%%%%%%%%%%%%%%%%%%%%%%%%%%%%%%%%%%%%%%%%%%%
%Landau notation
%\newcommand{\smallo}[1]{o\left(#1\right)}
%\newcommand{\bigo}[1]{O\left(#1\right)}
%\newcommand{\smallomega}[1]{\omega\left(#1\right)}
%\newcommand{\Th}[1]{\Theta\left(#1\right)}

%%%%%%%%%%%%%%%%%%%%%%%%%%%%%%%%%%%%%%%%%%%%%%%%%%%%%%%
%Graph classes and random graphs

\def\planargraph{P} %planar graph
\def\planarclass{\mathcal{P}} %class of planar graphs

\def\multigraph{M} %multigraph

\def\forest{F} %forest of trees with specified roots
\def\forestclass{\mathcal{F}} %class of forests of trees with specified roots

\def\nocomplex{U} %graph without complex component
\def\nocomplexclass{\mathcal{U}} %class of graphs without complex components

%%%%%%%%%%%%%%%%%%%%%%%%%%%%%%%%%%%%%%%%%%%%%%%%%%%%%%%
%Graph notations
%\newcommand{\maxdegree}[1]{\Delta\left(#1\right)} %maximum degree
\newcommand{\maxdegreeLarge}[1]{\Delta\big(#1\big)} %maximum degree (with big parentheses)
\newcommand{\largestcomponent}[1]{L\left(#1\right)} %largest component
\def\Largestcomponent{L} %largest component
%\newcommand{\rest}[1]{R\left(#1\right)} %graph without largest component ('rest')
%\def\Rest{R} %graph without largest component ('rest')
\def\degreesequence{\mathbf{d}}
\def\ratio{\mu}
\def\niv{k}
\def\nie{l}
\def\md{d}
\newcommand{\degree}[2]{d_{#2}\left(#1\right)} %degree of a vertex
%\newcommand{\vertexSet}[1]{V\left(#1\right)} %vertex set of a graph
%\newcommand{\edgeSet}[1]{E\left(#1\right)} %edge set of a graph
%\newcommand{\numberVertices}[1]{v\left({#1}\right)} %number of vertices in a graph
%\newcommand{\numberEdges}[1]{e\left({#1}\right)} %number of edges in a graph
\newcommand{\numberVerticesLarge}[1]{v\big({#1}\big)} %number of vertices in a graph (with big parentheses)
\newcommand{\numberEdgesLarge}[1]{e\big({#1}\big)} %number of edges in a graph (with big parentheses)


%%%%%%%%%%%%%%%%%%%%%%%%%%%%%%%%%%%%%%%%%%%%%%%%%%%%%%%
%Decomposition
%\newcommand{\core}[1]{C\left(#1\right)} %core
\newcommand{\complexpart}[1]{Q\left(#1\right)} %complex part
\newcommand{\restcomplex}[1]{U\left(#1\right)} %graph without complex part
\newcommand{\restcomplexLarge}[1]{U\big(#1\big)}
\def\Restcomplex{U} %graph without complex part
\def\Complexlargestcore{Q_L} %complex part of the largest component of the core
\newcommand{\complexlargestcore}[1]{Q_L\left(#1\right)}
\newcommand{\complexlargestcoreLarge}[1]{Q_L\big(#1\big)}
\def\Complexrestcore{Q_S} %complex part of the rest (=graph without largest component) of the core
\newcommand{\complexrestcore}[1]{Q_S\left(#1\right)}
\newcommand{\complexrestcoreLarge}[1]{Q_S\big(#1\big)}
\def\complexclass{\mathcal{Q}}
\def\complexgraph{Q}
\def\funcL{\ell}
\def\funcR{r}

%%%%%%%%%%%%%%%%%%%%%%%%%%%%%%%%%%%%%%%%%%%%%%%%%%%%%%%
%conditional random graphs
\def\cl{\mathcal{A}} %class of graphs
\def\func{\Phi} %function with domain a set of graphs
%\def\seq{\mathbf{a}} %sequence
\def\term{a} %element of the sequence
%newcommand{\condGraph}[2]{#1 \mid #2} %conditional random graph
%\def\randomGraph{A}
%\def\property{\mathcal{R}}

%%%%%%%%%%%%%%%%%%%%%%%%%%%%%%%%%%%%%%%%%%%%%%%%%%%%%%%
%Notation for random variables
\newcommand{\contiguous}[2]{#1 \triangleleft #2} %contiguous random variables

%%%%%%%%%%%%%%%%%%%%%%%%%%%%%%%%%%%%%%%%%%%%%%%%%%%%%%%
%Pruefer sequence
\def\prueferseqence{\psi} %Pruefer sequence
\def\prueferinvers{\psi^{-1}} %invers function of the Pruefer sequence
\newcommand{\sequences}[2]{\mathcal{S}\left(#1, #2\right)} %set of Pruefer sequences
\newcommand{\frequency}[2]{\#\left(#1, #2\right)} %number of occurences of an element in a Pruefer sequence


\newcommand{\setbuilder}[2]{\left\{#1 \mid #2\right\}} %set-builder notation

\newcommand{\lessorequal}{\hspace{0.06cm}\leq\hspace{0.06cm}}
\newcommand{\greaterorequal}{\hspace{0.06cm}\geq\hspace{0.06cm}}
\newcommand{\equal}{\hspace{0.06cm}=\hspace{0.06cm}}
\newcommand{\greater}{\hspace{0.06cm}>\hspace{0.06cm}}
\newcommand{\defined}{\hspace{0.06cm}:=\hspace{0.06cm}}

\newcommand{\transformation}[2]{#1\rightarrow #2}

\newtheorem{question}[thm]{Question}
\crefname{question}{question}{questions}


%---Macros-Process------------------------------------------------------------
%\newcommand{\whp}{whp}%with high probability}
%\newcommand{\Whp}{Whp}%Beginning of a sentence
%\newcommand{\prob}[1]{\mathbb{P}\left[#1\right]} %Probability
%\newcommand{\probLarge}[1]{\mathbb{P}\big[#1\big]} %Probability with large parentheses
%\newcommand{\condprob}[2]{\mathbb{P}\left[#1 \;\middle|\; #2\right]}
%\newcommand{\variance}[1]{\mathbb{V}\left[#1\right]}
%\newcommand{\expec}[1]{\mathbb{E}\left[#1\right]} %Expectation

%\def\Erdos{Erd\H{o}s}
%\def\Renyi{R\'enyi}
%\def\Luczak{\L{}uczak}
%\def\ER{\Erdos-\Renyi}
%\def\Bollobas{Bollob\'{a}s}

%\newcommand{\proofof}[1]{\subsection{Proof of \Cref{#1}}}
\newcommand{\proofofW}[1]{\subsection*{Proof of \Cref{#1}}}

%\def\N{\mathbb{N}} %natural numbers
%\def\R{\mathbb{R}} %real numbers
%\def\ur{\in_R} %element chosen uniformly at random
%\newcommand{\rounddown}[1]{\left\lfloor#1\right\rfloor} %floor function

%%%%%%%%%%%%%%%%%%%%%%%%%%%%%%%%%%%%%%%%%%%%%%%%%%%%%%%
%Landau notation
%\newcommand{\smallo}[1]{o\left(#1\right)}
%\newcommand{\bigo}[1]{O\left(#1\right)}
%\newcommand{\smallomega}[1]{\omega\left(#1\right)}
%\newcommand{\Th}[1]{\Theta\left(#1\right)}

%%%%%%%%%%%%%%%%%%%%%%%%%%%%%%%%%%%%%%%%%%%%%%%%%%%%%%%
%Graph notations
%\newcommand{\maxdegree}[1]{\Delta\left(#1\right)} %maximum degree
%\newcommand{\maxdegreeLarge}[1]{\Delta\big(#1\big)} %maximum degree (with big parentheses)
%\newcommand{\largestcomponent}[1]{L\left(#1\right)} %largest component
%\def\Largestcomponent{L} %largest component
%\newcommand{\rest}[1]{R\left(#1\right)} %graph without largest component ('rest')
%\def\Rest{R} %graph without largest component ('rest')

%\newcommand{\vertexSet}[1]{V\left(#1\right)} %vertex set of a graph
%\newcommand{\edgeSet}[1]{E\left(#1\right)} %edge set of a graph
%\newcommand{\numberVertices}[1]{v\left({#1}\right)} %number of vertices in a graph
%\newcommand{\numberEdges}[1]{e\left({#1}\right)} %number of edges in a graph
%\newcommand{\numberVerticesLarge}[1]{v\big({#1}\big)} %number of vertices in a graph (with big parentheses)
%\newcommand{\numberEdgesLarge}[1]{e\big({#1}\big)} %number of edges in a graph (with big parentheses)

%\newcommand{\setbuilder}[2]{\left\{#1 \mid #2\right\}} %set-builder notation

%\def\cl{\mathcal{P}}
%\newcommand{\pro}[2]{P_{#1,#2}} %graph process (graph after #2 edges have been tested)
%\newcommand{\acc}[2]{P_{#1,m=#2}} %graph process (graph after #2 edges have been accepted)
%\newcommand{\er}[2]{G_{#1,#2}} %graph process (graph after #2 edges have been considered and all of them were added)

\newcommand{\sol}[1]{\beta\left(#1\right)}
\newcommand{\funcPro}[1]{f\left(#1\right)}
\newcommand{\invFunc}[1]{f^{-1}\left(#1\right)}

\DeclareMathOperator{\excess}{ex}
\newcommand{\ex}[1]{\excess\left(#1\right)}
%\newcommand{\weight}[1]{w\left(#1\right)}
%\newcommand{\core}[1]{C\left(#1\right)} %core
\newcommand{\forestPro}[1]{F\left(#1\right)} 
\DeclareMathOperator{\binDistribution}{Bin}
\newcommand{\binPro}[2]{\binDistribution\left(#1,#2\right)} 
\DeclareMathOperator{\numberTreeComponents}{\#t}
\newcommand{\nt}[1]{\numberTreeComponents\left(#1\right)} 
\DeclareMathOperator{\considered}{q}
\newcommand{\con}[1]{\considered\left(#1\right)}
\DeclareMathOperator{\forbidden}{forb}
\newcommand{\forb}[1]{\forbidden\left(#1\right)}
\DeclareMathOperator{\addable}{add}
\newcommand{\add}[1]{\addable\left(#1\right)}
\newcommand{\rej}[1]{r\left(#1\right)}


%---Macros-Local-structure------------------------------------------------------------
%\newcommand{\class}[1]{\mathcal{#1}}
%\newcommand{\whp}{whp}%with high probability}
%\newcommand{\Whp}{Whp}%Beginning of a sentence
%\newcommand{\prob}[1]{\mathbb{P}\left[#1\right]}    %Probability
%\newcommand{\probLarge}[1]{\mathbb{P}\big[#1\big]}    %Probability with large parentheses
%\newcommand{\condprob}[2]{\mathbb{P}\left[#1 \;\middle|\; #2\right]}
%\newcommand{\variance}[1]{\mathbb{V}\left[#1\right]}
%\newcommand{\expec}[1]{\mathbb{E}\left[#1\right]}    %Expectation
%\DeclareMathOperator{\po}{Po}
%\newcommand{\poisson}[1]{\po\left(#1\right)}


%\def\Erdos{Erd\H{o}s}
%\def\Renyi{R\'enyi}
%\def\Luczak{\L{}uczak}
%\def\ER{\Erdos--\Renyi}
%\def\Bollobas{Bollob\'{a}s}
%\def\GW{Galton--Watson}



%\def\N{\mathbb{N}} %natural numbers
%\def\Z{\mathbb{Z}} %integers
%\def\R{\mathbb{R}} %real numbers
%\def\ur{\in_R} %element chosen uniformly at random
\newcommand{\roundup}[1]{\left\lceil#1\right\rceil} %ceil function
%\newcommand{\rounddown}[1]{\left\lfloor#1\right\rfloor} %floor function



%%%%%%%%%%%%%%%%%%%%%%%%%%%%%%%%%%%%%%%%%%%%%%%%%%%%%%%
%random forests
\def\forest{F}
\def\forestClass{\mathcal{F}}

%%%%%%%%%%%%%%%%%%%%%%%%%%%%%%%%%%%%%%%%%%%%%%%%%%%%%%%
%Landau notation
%\newcommand{\smallo}[1]{o\left(#1\right)}
%\newcommand{\bigo}[1]{O\left(#1\right)}
%\newcommand{\smallomega}[1]{\omega\left(#1\right)}
%\newcommand{\Op}[1]{O_p\left(#1\right)}
%\newcommand{\Th}[1]{\Theta\left(#1\right)}
%\newcommand{\ThetaP}[1]{\Theta_p\left(#1\right)}
%%%%%%%%%%%%%%%%%%%%%%%%%%%%%%%%%%%%%%%%%%%%%%%%%%%%%%%
%Graph classes
%\def\planargraph{P}     %planar graph
%\def\planarclass{\mathcal{P}} %class of planar graphs
\def\erClass{\mathcal{G}} %class of graphs
\def\nocomplexClass{\mathcal{U}} %class of graphs
%%%%%%%%%%%%%%%%%%%%%%%%%%%%%%%%%%%%%%%%%%%%%%%%%%%%%%%
%Graph notations
%\newcommand{\largestcomponent}[1]{L\left(#1\right)} %largest component
%\def\Largestcomponent{L} %largest component
\newcommand{\restLocal}[1]{S\left(#1\right)} %graph without largest component ('rest')
\def\RestLocal{S} %graph without largest component ('rest')
%\newcommand{\vertexSet}[1]{V\left(#1\right)} %vertex set of a graph
%\newcommand{\edgeSet}[1]{E\left(#1\right)} %edge set of a graph
%\newcommand{\numberVertices}[1]{v\left({#1}\right)} %number of vertices in a graph
%\newcommand{\numberEdges}[1]{e\left({#1}\right)} %number of edges in a graph
\DeclareMathOperator{\dist}{dist}
\newcommand{\distance}[3]{\dist_{#1}\left(#2, #3\right)}
%\newcommand{\degree}[2]{d_{#2}\left(#1\right)} %degree of a vertex

%%%%%%%%%%%%%%%%%%%%%%%%%%%%%%%%%%%%%%%%%%%%%%%%%%%%%%%
%Core-Kernel approach
%\newcommand{\core}[1]{C\left(#1\right)} %core
%\newcommand{\kernel}[1]{K\left(#1\right)} %kernel
%\newcommand{\complexpart}[1]{Q\left(#1\right)} %complex part
%\newcommand{\restcomplex}[1]{U\left(#1\right)} %graph without complex part
%\newcommand{\restcomplexLarge}[1]{U\big(#1\big)}
%\def\Restcomplex{U} %graph without complex part
%\def\complexclass{\mathcal{Q}}
%\def\complexgraph{Q}
\def\numberVerticesOutside{n_U}
\def\numberEdgesOutside{m_U}
\newcommand{\subdivisionNumberLocal}[2]{S_{#1}\left(#2\right)}
%%%%%%%%%%%%%%%%%%%%%%%%%%%%%%%%%%%%%%%%%%%%%%%%%%%%%%%
%conditional random graphs
%\def\cl{\mathcal{A}} %class of graphs
%\def\func{\Phi} %function with domain a set of graphs
\def\seqLocal{\mathbf{a}} %sequence
%\def\term{a} %element of the sequence
%\newcommand{\condGraph}[2]{#1 \mid #2} %conditional random graph
\def\imageSet{\mathcal{S}} 

%%%%%%%%%%%%%%%%%%%%%%%%%%%%%%%%%%%%%%%%%%%%%%%%%%%%%%%
%rooted graphs and limits
\def\root{r} %root of a graph
\def\rootSet{R} %root of a graph
\def\radius{\ell} %root of a graph
%\newcommand{\ball}[3]{B_{#1}\left(#2, #3\right)}
\newcommand{\ballP}[2]{B_{#1}\left(#2\right)}
\def\isomorphic{\cong}
%\newcommand{\rootedGraph}[2]{\left(#1, #2\right)}
%\newcommand{\distributionalLimit}[2]{#1~\xrightarrow{~d~}~#2}
%\DeclareMathOperator{\GWT}{GWT}
%\newcommand{\gwt}[1]{\GWT\left(#1\right)}
%\def\skeletonTree{T_\infty}
\newcommand{\skeletonTreeRays}[1]{T_\infty^{\left(#1\right)}}
\newcommand{\unbiased}[1]{#1_\circ}
\newcommand{\contiguousLocal}[2]{#1 \triangleleft \triangleright #2}
\def\detGraph{W} %a deterministic rooted graph
\def\planeTree{Z} %a deterministic plane tree
\newcommand{\infinitepath}[1]{P_{\infty}^{\left(#1\right)}}
%%%%%%%%%%%%%%%%%%%%%%%%%%%%%%%%%%%%%%%%%%%%%%%%%%%%%%%
%\newcommand{\setbuilder}[2]{\left\{#1 \mid #2\right\}} %set-builder notation
\newcommand{\setbuilderBig}[2]{\big\{#1 \mid #2\big\}}
%\newcommand{\lessorequal}{\hspace{0.06cm}\leq\hspace{0.06cm}}
%\newcommand{\greaterorequal}{\hspace{0.06cm}\geq\hspace{0.06cm}}
%\newcommand{\equal}{\hspace{0.06cm}=\hspace{0.06cm}}
%\newcommand{\greater}{\hspace{0.06cm}>\hspace{0.06cm}}
%\newcommand{\defined}{\hspace{0.06cm}:=\hspace{0.06cm}}
\newcommand{\symmetricDifference}[2]{#1\triangle#2}
%\def\ntrees{t} %number of trees